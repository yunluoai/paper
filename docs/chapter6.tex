\chapter{总结与展望}


%*********************************************************************
% 6.1 论文总结
%*********************************************************************
\section{论文总结}
本文围绕DSP架构下基于LLVM的GlobalISel框架的工程化实现与优化问题,系统开展了架构分析、优化设计、实现验证以及性能评估等研究工作,形成了一套面向DSP后端的全局指令选择实现与测试评估方案。全文的主要研究内容与成果可以总结为以下几个方面。

\begin{enumerate}
	
	\item
	首先,在指令选择框架层面,本文对GlobalISel的整体架构与关键阶段进行了系统分析,重点围绕GMIR生成、指令合法化、寄存器组选择以及机器指令选择等核心流程,结合DSP架构特性,完成了全局指令选择在DSP后端中的功能性实现。通过对基本算术指令、控制流指令以及函数调用等典型场景的端到端验证,证明了GlobalISel框架在DSP后端中的可行性与正确性,为后续优化工作奠定了基础。
	
	\item
	其次,在优化策略设计方面,本文针对GlobalISel在不同阶段产生的冗余与性能问题,提出了分阶段的指令选择优化思路,并围绕内存操作指令、乘法指令以及其他通用指令模式,设计并实现了多项具有针对性的优化策略。其中,合法化前的通用合并优化侧重于消除表达式级冗余和规范化指令形态,而合法化后的目标相关优化则更加关注DSP专用指令匹配、代码密度提升以及寄存器使用效率。实验结果表明,这些优化能够在保证语义正确性的前提下,有效改善生成代码的质量。
	
	\item
	再次,在测试与评估体系方面,本文设计并实现了一套面向DSP编译器的测试评估平台。该平台以随机测试生成技术为基础,结合DSP架构特点对YARPGen进行定制化扩展,构建了覆盖指令选择与后端优化场景的随机测试输入体系。在此基础上,平台实现了从测试执行、数据采集到结果分析与可视化展示的完整流程,能够支持跨版本性能对比、历史趋势分析以及回归问题定位,为编译器优化提供了可靠的实验支撑。
	
	\item
	最后,通过在Embench等测试集上的系统实验,本文从编译时间、代码尺寸和执行周期等多个维度,对比分析了基于DAG的指令选择方案、未优化的GlobalISel以及优化后的GlobalISel实现。实验结果表明,在DSP架构下,GlobalISel在编译效率方面具有一定优势,结合本文提出的优化策略后,在代码尺寸和执行效率等方面亦表现出较为稳定的改进效果,验证了本文方法在工程实践中的有效性。
	
\end{enumerate}

综上所述,本文围绕DSP后端的GlobalISel指令选择实现与优化问题,完成了从架构分析、优化设计到测试验证的系统性研究,为后续在DSP平台上进一步推进GlobalISel的应用和优化提供了实践基础。


%*********************************************************************
% 6.2 论文展望
%*********************************************************************
\section{论文展望}
尽管本文在DSP架构下对GlobalISel指令选择及其优化进行了较为系统的研究,但仍存在进一步拓展和深化的空间,后续工作可从以下几个方向展开:

\begin{enumerate}
	
	\item
	在指令选择优化层面,本文当前的优化策略主要集中在局部模式合并和指令级重写,尚未充分结合更高层次的全局信息。未来可以进一步探索将数据流分析、控制流分析与GlobalISel的合并优化机制相结合,在函数级甚至跨基本块层面挖掘更具潜力的优化机会,以进一步提升代码质量和执行效率。

	\item
	在DSP架构特性利用方面,本文的优化设计主要围绕通用算术指令和部分专用指令展开。随着DSP指令集和微架构特性的不断演进,未来可进一步针对流水线结构、指令并行执行能力以及专用加速单元,设计更细粒度的目标相关优化策略,使GlobalISel在DSP平台上的优势得到更充分发挥。

	\item
	在测试评估体系方面,当前平台主要侧重于性能指标的离线统计与可视化分析。后续工作可以考虑引入更自动化的性能回归检测机制,例如基于阈值或趋势分析的自动告警策略,以提升平台在持续集成和日常开发中的实用性。同时,也可扩展测试输入类型,引入更多面向实际应用场景的负载程序,以增强测试结果的代表性。

	\item
	从整体工具链角度来看,GlobalISel仍处于持续演进阶段。未来可结合LLVM社区在GlobalISel方向上的最新进展,对本文实现进行持续迭代,并探索将相关优化经验推广至其他嵌入式或专用处理器架构,为构建统一、高效的指令选择框架提供更多实践参考。

\end{enumerate}








