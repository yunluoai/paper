\chapter{总结与展望}


%*********************************************************************
% 6.1 论文总结
%*********************************************************************
\section{论文总结}
本文围绕DSP架构下基于LLVM的GlobalISel框架的工程化实现与优化问题,系统开展了架构分析、优化设计、实现验证以及性能评估等研究工作,形成了一套面向DSP后端的全局指令选择实现与测试评估方案。全文的主要研究内容与成果可以总结为以下几个方面。

\par

第一,本文针对GMIR生成、指令合法化、寄存器组选择以及机器指令选择等核心阶段来对GlobalISel进行系统的分析,结合DSP架构的特性,在DSP后端上完成了GlobalISel中的实现。通过对基本算术指令及控制流指令这类典型场景的验证,证明了GlobalISel框架在DSP后端中实现的可行性与正确性,为后续的优化工作奠定了基础。

\par

第二,本文针对GlobalISel不同阶段产生的指令冗余所引起的性能问题,提出了分阶段的指令选择优化思路,并围绕内存操作指令、乘法指令以及其他通用指令模式,设计并实现了多项具有针对性的优化策略。其中合法化前的通用合并优化侧重于消除表达式级冗余和规范化指令形态,而合法化后的目标相关优化则更加关注DSP专用指令匹配、代码密度提升以及寄存器使用效率。

\par

第三,本文针对DSP现有测试体系存在的问题,设计并实现了一套完备的测试评估平台。平台由测试子系统和评估子系统构成,测试子系统以随机测试生成技术为基础,结合DSP架构特性对YARPGen进行定制化扩展,提高了测试的覆盖率。评估子系统用于整理并可视化展示采集到的测试数据,子系统支持跨版本性能对比、历史趋势分析以及回归问题定位等功能,为编译器优化提供了可靠的数据支撑。

\par

第四,本文从编译时间、代码体积和执行周期等多个维度,对基于DAG的指令选择方案、未优化的GlobalISel以及优化后的GlobalISel进行了对比分析。实验结果表明,在DSP架构下,GlobalISel在编译效率方面具有一定优势;在引入本文提出的优化策略后,其生成代码在体积和执行效率等方面均表现出较为稳定的改进效果,从而验证了所提出方法在工程实践中的有效性。


%*********************************************************************
% 6.2 论文展望
%*********************************************************************
\section{论文展望}
本文在DSP架构下对GlobalISel的实现和优化进行了系统的研究,并针对DSP现有测试体系中存在的问题,设计并实现了一套测试评估平台。尽管这些工作在实践中取得了初步成效,但在优化深度与覆盖范围等方面仍存在进一步提升的空间,后续工作可从以下几个方向进行展开。

\par

首先,本文当前实现的优化策略主要集中在局部模式合并和指令级重写,并没有充分利用到更高层次的全局信息。未来可以将数据流分析、控制流分析与GlobalISel的合并优化机制相结合,在函数级层面挖掘更多的优化机会,用于进一步提升代码质量和执行效率。

\par

其次,本文的优化设计主要围绕通用算术指令和部分专用指令展开,未来可以进一步针对流水线结构、指令并行执行能力以及专用加速单元,设计更细粒度的目标相关优化策略,使GlobalISel在DSP平台上的优势得到更充分发挥。

\par

最后,当前平台的重心在于对性能指标进行离线统计分析与可视化呈现。为增强其在持续集成与日常开发流程中的实用价值,后续可考虑引入更为自动化的性能回归检测机制。例如,可建立基于预设性能阈值的自动判定策略,或采用对性能数据趋势进行智能分析的告警模型,从而实现对性能退化的早期发现与主动预警。
