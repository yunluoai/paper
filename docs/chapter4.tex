\chapter{测试评估平台设计与实现}


%*********************************************************************
% 4.1 DSP编译器测试现状与改进
%*********************************************************************
\section{DSP编译器测试现状与改进}
随着DSP编译器后端功能的不断扩展,特别是全局指令选择框架以及全局指令选择的一些优化方法的引入,编译器在指令匹配、寄存器组的选择、合法化等各阶段上的行为对输入越加敏感,以及组合关系也越加复杂。尽管全局指令选择在实现上比基于DAG的指令选择方法更加模块化,但其指令选择结果由多个Pass的协同决策共同决定,这导致整个过程对输入的程序结构更加敏感。

\par

DSP工具链已经建立了较为完整的CI(Continuous Integration, 持续集成)测试流程,能够较好地保证主分支代码的功能正确性。但是该CI流程主要面向传统编译流程构建,其设计目的是保证功能的可用性,没有针对全局指令选择及其对应优化在整个复杂流程中的正确性和性能稳定性做到有针对性的保证。随着GlobalISel在DSP工具链中的落地,现有CI测试在测试结果可分析性、性能稳定性判断、复杂输入场景覆盖上暴露出问题,需要对相关问题进行总结分析。


% 4.1.1 测试流程现状
\subsection{测试流程现状}
对于现有的测试流程,开发者提交代码后,CI系统会自动触发预定义的测试流程,来对最新的工具链代码进行编译和测试验证。只有在所有测试通过且获得项目负责人批准后,相关代码才会被合并到主分支。这一测试流程在当前开发阶段发挥着重要作用:一方面,它能在代码提交阶段就发现编译失败、链接错误或明显的功能缺陷,从而确保主分支代码的基本可用性;另一方面,统一的CI流水线使代码提交流程标准化,减少了人工测试的需求,进而降低了开发成本。

\par

总体而言,现有的测试体系具备验证功能正确性所需的基本能力,能够有效发现明显错误并阻止重大缺陷侵入主分支。然而,该体系主要以测试通过与否作为评估标准,缺乏对编译器优化效果、性能变化以及复杂场景下系统鲁棒性的系统性支持,难以满足DSP工具链对高性能和高可靠性长期发展的需求。具体而言,测试用例覆盖率不足以及测试结果的可读性和分析能力较弱是两个关键问题。


\subsubsection{1.测试集来源单一,场景覆盖局限于常见路径,缺乏随机化测试机制}
测试用例的覆盖完整性直接决定了DSP工具链发现潜在缺陷的能力。然而,现有DSP工具链在测试集构成与场景覆盖方面仍存在明显不足,难以对核心编译流程的鲁棒性进行充分验证。从测试来源来看,当前DSP工具链的测试集主要包括以下三类:

\begin{itemize}
	\item
	基于库函数的手写测试用例:围绕基础功能验证设计,重点覆盖C标准库、数学库以及运行时支持库等核心库函数的调用场景。
	
	\item
	移植的开源Benchmark程序:以通用计算或嵌入式系统性能评估为目标(如Embench),用于衡量工具链在典型应用负载下的性能表现。
	
	\item 
	已知缺陷的最小复现样例:开发过程中针对已暴露的问题提炼出的最小测试用例,用于回归测试,防止同类缺陷再次引入。
	
\end{itemize}

上述测试用例在验证工具链基础功能正确性以及复现已知缺陷方面具有一定价值,但其整体覆盖范围仍然较为有限。这是因为测试用例的设计很大程度上依赖于人工经验,往往集中于库函数的常规调用路径和标准输入输出场景,对于极端数据类型、异常输入以及复杂控制流等边界条件覆盖不足。

\par

这种覆盖不足会直接导致两类风险。第一,工具链在通用应用环境中可能运行正常,但在特定的应用场景中却容易出现逻辑错误。第二,编译器内部条件引发的缺陷往往难以提前检测。这类错误通常仅出现在特定的指令组合或数据模式下,例如指令选择阶段的模式识别错误、寄存器分配中的资源冲突,或是优化转换过程中产生的语义偏差。如果这些问题在测试阶段未能被发现,就可能在实际运行环境中引发崩溃或性能异常,造成严重后果。

\par

对于引入GlobalISel后的DSP编译流程而言,上述问题进一步被放大。由于指令选择结果往往由多个Pass的协同决策共同决定,例如寄存器组选择与指令合法化之间的相互影响、不同合法化路径对后续指令匹配结果的约束关系等,这类行为通常只会在特定输入结构和数据特征组合下才会显现。依赖人工经验构造的测试样例难以覆盖这些复杂交互场景,导致部分潜在缺陷在早期测试阶段无法暴露出来,直至在实际的DSP应用中才被触发,从而引发较高的工程风险。


\subsubsection{2.测试产物可读性与分析能力不足}
现有测试体系在测试结果的表达和分析能力方面同样存在明显不足。目前的CI流程只是把测试用例是否通过作为判定标准,产出的测试结果缺乏对编译器后端行为的量化描述,未系统记录程序执行周期、代码尺寸、指令打包效率等关键性能指标,也未对生成的汇编代码进行结构化管理。这使得测试结果只能用于验证功能正确性,而无法反映全局指令选择及相关优化策略对编译器性能造成的实际影响。

\par

除此之外,测试结果既没有做历史化存储,也没有做跨版本管理,这导致开发人员难以对不同提交之间的性能变化进行对比分析,也无法从整体上观察工具链性能随时间演进的趋势。随着CI测试记录不断积累,大量中间提交和实验性版本产生的结果进一步加剧了测试数据杂乱的问题,显著提高了有效信息筛选和问题定位的难度和成本。

\par

综上所述,现有测试体系在输入覆盖能力和结果分析深度方面存在明显局限,难以支撑全局指令选择框架下编译器后端行为的系统性验证,已无法满足DSP工具链在高性能和高可靠性方向持续演进的需求。


% 4.1.2 测试体系优化需求与技术路径
\subsection{测试体系优化需求与技术路径}
基于4.1.1节的分析可知,随着全局指令选择框架及相关优化策略在DSP工具链中的实现,单纯依赖人工构造测试样例,以样例通过与否这一单一判定标准,难以支撑对编译器后端行为的系统性验证。为了确保全局指令选择在复杂场景下的功能与性能的稳定,现有测试体系需从测试输入覆盖与测试结果分析深度两方面进行针对性优化。

\par

为应对测试输入覆盖不足的问题,现有测试体系需提升测试用例的覆盖广度与复杂场景的模拟能力,以充分验证全局指令选择框架中基于条件触发的各类行为。由于指令选择、寄存器组绑定和合法化等阶段的决策往往依赖于输入程序的控制流结构、数据分布以及运算模式,仅仅依靠人工设计的测试样例难以系统覆盖所有潜在组合路径。因此,有必要引入随机测试生成机制,通过自动化方式生成大量具有结构多样性和数据随机性的测试程序,从而覆盖人工测试难以触及的边缘场景。这种随机测试方式能够有效暴露隐藏在特定输入条件下的潜在缺陷,提升编译器后端对复杂输入场景的整体鲁棒性。

\par

对于测试结果分析深度不足的问题,仅验证程序功能是否正确已无法满足全局指令选择优化验证的实际需求。全局指令选择及其相关优化策略的引入,往往不会直接导致功能错误,而是可能以性能退化、代码尺寸膨胀或指令级行为异常等形式体现出来。为准确评估其对DSP架构性能的影响,有必要构建一套能够自动采集、存储和分析性能指标的评估平台,对程序执行周期、代码尺寸及指令级行为进行量化记录,并支持跨版本的对比分析和趋势观察。通过将测试结果结构化存储并以可视化方式呈现,可有效降低性能分析门槛,辅助开发者快速定位性能回退和异常变化。

\par

综上所述,面向全局指令选择框架的DSP工具链测试体系,应当以随机测试驱动的高覆盖输入验证为基础,以性能指标采集与可视化分析为核心支撑,构建一套覆盖广、可量化、可追溯的测试评估机制。针对这一目标,本文在后续章节中设计并实现了一套集随机测试生成与性能评估于一体的测试评估平台,该平台可有效支撑全局指令选择及其优化策略的功能验证与性能量化分析,为技术方案的落地提供系统化保障。


% 4.1.3 随机测试生成技术与生成器选型
\subsection{随机测试生成技术与生成器选型}
随机测试生成技术是自动化测试的核心技术之一,其核心原理是使用随机或伪随机算法,按照预设的语法和语义规则生成大规模的测试输入,以此验证软件系统在不同输入场景下功能的正确性、运行的稳定性以及整体的鲁棒性。和传统的人工编写测试用例的方式相比,随机测试能够在更短时间内覆盖更广的输入空间,有利于发现隐藏较深、难以通过人工经验判断的边界错误和组合缺陷。

\par

在编译器测试中,随机测试技术通常借助专用的随机测试生成器来实现。这类工具会随机组合表达式、控制流结构、数据类型以及运算符等语言构造要素,来生成大量结构各异的测试程序,并将其作为输入交由编译器处理,从而验证编译器前端解析、优化过程以及后端代码生成等多个阶段的正确性与健壮性。

\par

在编译器随机测试工具的发展进程中,CSmith是较早在业界得到广泛应用的C语言随机程序生成器,其目标是发现编译器在处理常规C程序时可能存在的逻辑错误。CSmith的特点是生成的随机程序不仅严格遵循C99标准,还会主动规避UB(Undefined Behavior, 未定义行为),从而保证测试结果可复现。

\par

CSmith在验证编译器的基础语义一致性、防范功能性错误等方面起到了关键作用,但受限于设计目标,它的测试重心集中于功能正确性的校验,对于编译器在高性能优化场景下的行为,覆盖能力存在明显局限。

\par

随着现代处理器架构日趋复杂,编译器的优化策略也在持续迭代升级,仅依靠功能正确性验证已难以满足编译器工具链的测试需求。在这一背景下,YARPGen(Yet Another Random Program Generator)应运而生,它的设计初衷正是针对CSmith在性能相关测试方面覆盖不足的问题,重点面向编译器中后端的性能优化阶段开展随机化测试。它可以在不依靠动态检查的情况下,使用生成规则来生成没有未定义行为的表达式,同时提升了生成代码的多样性,触发更多编译器优化的可能。在代码生成策略上,YARPGen侧重构造密集型循环、数组连续访问、数据并行计算与向量运算等典型代码模式,从而有针对性地覆盖循环向量化、指令级并行、缓存优化以及指令调度等编译器后端的关键优化阶段。

\par

从语言兼容和架构适配上来看,YARPGen不仅原生支持C语言,还支持C++的部分核心特性,并能够生成面向SIMD、向量指令等现代处理器架构扩展的针对性代码。这一特性使其在测试面向高性能计算的嵌入式DSP场景时具有一定的实用价值。

\par

基于上述随机测试生成器的特点,本文在测试体系的设计中,选择了YARPGen并结合DSP架构特性对其进行定制化扩展。通过生成含DSP内建函数集的随机测试程序,测试体系能更高效地验证编译器行为,为后续性能评估与回归分析提供可靠的测试基础。


%*********************************************************************
% 4.2 平台设计
%*********************************************************************
\section{平台设计}


% 4.2.1 平台整体设计
\subsection{平台整体设计}
为解决4.1.1节提到的测试覆盖不足、结果分析能力弱等问题,本文设计并实现了一套面向DSP编译器的性能测试评估平台。平台整体分为测试子系统与评估子系统两大模块,前者在原有CI测试平台上进行功能新增与架构重构,后者则为全新搭建的独立模块。

\par

测试子系统的核心改进在于移植并定制随机测试生成器YARPGen,拓展其功能,添加适配DSP指令集架构的Intrinsic(内建函数)集,同时完成适配现有CI流程的环境配置。该子系统在每次测试触发时可自动生成一批随机样例,并与原有固定测试集并行执行,同时将测试过程中检出错误的样例纳入固定测试集,形成迭代优化的测试用例库。

\par

评估子系统对接线上CI测试平台,通过脚本实现编译流程的自动化触发。在开发人员提交PR后,脚本会自动触发测试任务,同时在部署在Runner上的仿真器和芯片上执行;测试完成后,结果会同步至Gitlab的Perf仓库;后端系统定时拉取Perf仓库并更新数据库,当前端发起请求时,系统会返回查询结果并在前端进行可视化展示,以此直观地呈现新变更引入的性能提升或退步。

\par

系统整体设计如图\ref{fig:system_flowchart}所示,整体架构分为测试子系统和评估子系统两部分。测试子系统完成从测试样例生成、编译到执行的全自动化流程;评估子系统则通过前端、后端与数据库的协作,实现测试结果的结构化存储、查询、对比分析与可视化展示。两个子系统功能上解耦,通过Perf仓库实现数据交互。这种设计既保证了测试流程的高效运行,又提升了测试结果的可分析性与可追溯性。

\begin{figure}[htbp]
	\centering
	\includegraphics[width=1\textwidth]{pics/system_flowchart.png}
	\caption{系统整体设计图}
	\label{fig:system_flowchart}
\end{figure}


% 4.2.2 测试子系统设计
\subsection{测试子系统设计}
测试子系统的核心目标是在CI环境下构建一套全自动化的测试流水线,覆盖从测试用例生成、编译执行到结果归档与分析的完整流程。该子系统不仅用于验证编译器功能正确性,还承担着对编译器性能变化进行长期、稳定跟踪的任务,为后续的性能评估与回归分析提供可靠数据基础。

\par

从功能上看,测试子系统主要由随机测试样例生成模块、编译与测试模块以及结果处理模块三部分构成,如图\ref{fig:test-subsystem}所示:

\begin{itemize}
	\item
	随机测试样例生成模块:随机测试工具用于自动生成适配DSP架构的随机测试样例,重点覆盖密集循环、向量运算等性能相关场景。随机测试工具在执行测试时再生成,并将错误的样例推送至测试仓库,与其他测试样例一起被统一管理。二者共同为CI测试提供稳定的用例来源。
	
	\item
	编译与测试模块:CI测试平台作为调度核心,从编译器代码仓库中拉取开发者提交的最新DSP工具链代码,并从测试仓库中获取待执行的测试脚本、测试样例以及需要的测试工具。首先进行编译,编译成功后将测试任务分发至Gitlab Runner,触发两类测试流程:一类是在DSP架构模拟器上执行的测试,用于快速验证功能正确性及基础性能表现;另一类是在实际DSP硬件芯片上执行的测试,用于发现硬件上的Bug,以及获取更加准确的性能指标。
	
	\item
	结果处理模块:测试完成后,CI测试平台会将所有测试结果统一归档,并推送至Perf仓库。归档内容不仅包括测试通过或失败状态,还涵盖性能指标数据以及生成的汇编代码,为后续评估分析提供完整、可靠的数据基础。除此之外,为了解决4.1.2中提到的冗余问题,设置过滤器来保证只有主分支的代码才会传到Perf仓库中。
	
\end{itemize}

\begin{figure}[htbp]
	\centering
	\resizebox{0.9\linewidth}{!}{
		\begin{tikzpicture}[
			node distance=1.6cm,
			box/.style={draw, rectangle, rounded corners, align=center, minimum width=3cm, minimum height=0.9cm},
			smallbox/.style={draw, rectangle, rounded corners, align=center, minimum width=2.8cm, minimum height=0.8cm},
			dashedbox/.style={draw, rectangle, dashed, rounded corners, inner sep=0.5cm}
			]
			
			%==================== 随机测试样例生成模块 ====================
			\node[box] (rand) {YARPGen};
			\node[box, below of=rand] (testrepo) {测试仓库};
			
			\node[dashedbox, fit=(rand)(testrepo), label=above:{随机测试样例生成模块}] (gen) {};
			
			\draw[->] (rand) -- node[right]{错误样例推送} (testrepo);
			
			%==================== 编译与测试模块 ====================
			\node[box, right=4cm of rand] (compilerepo) {编译器仓库};
			\node[box, below of=compilerepo] (ci) {CI测试平台};
			\node[box, below of=ci] (runner) {GitLab Runner};
			
			\node[smallbox, below left=0.8cm and -0.6cm of runner] (sim) {模拟器测试};
			\node[smallbox, below right=0.8cm and -0.6cm of runner] (chip) {片上测试};
			
			\node[dashedbox, fit=(compilerepo)(ci)(runner)(sim)(chip),
			label=above:{编译与测试模块}] (build) {};
			
			\draw[->] (testrepo) -- (ci);
			\draw[->] (compilerepo) -- (ci);
			\draw[->] (ci) -- (runner);
			\draw[->] (runner) -- (sim);
			\draw[->] (runner) -- (chip);
			
			%==================== 结果处理模块 ====================
			\node[box, right=4cm of ci] (archive) {测试结果归档};
			\node[box, below of=archive] (filter) {分支过滤器};
			\node[box, below of=filter] (perfrepo) {Perf仓库};
			
			\node[dashedbox, fit=(archive)(filter)(perfrepo),
			label=above:{结果处理模块}] (result) {};
			
			\draw[->] (ci) -- (archive);
			\draw[->] (archive) -- (filter);
			\draw[->] (filter) -- (perfrepo);
			
		\end{tikzpicture}
	}
	\caption{测试子系统整体结构与功能模块设计图}
	\label{fig:test-subsystem}
\end{figure}


% 4.2.3 评估子系统设计
\subsection{评估子系统设计}
评估子系统的任务是实现测试结果的高效利用与直观呈现,采用前端、后端与数据库协同工作的三层架构,如图\ref{fig:eval-subsystem}所示,实现对CI流水线输出性能数据的统一接入、持久化存储与多维度分析。其设计目标主要包括以下三点:第一,将分散的测试产出进行结构化处理,构建可高效检索的存储体系;第二,提供标准化、稳定可靠的数据查询接口,支持跨编译器版本的性能历史对比与长期趋势分析;第三,将分析结果以交互式可视化形式呈现,从而提升性能回归问题的定位精度与诊断效率。系统的具体模块与流程如下:

\begin{itemize}
	\item
	后端模块负责从Perf仓库拉取性能报告,对原始数据进行结构化解析并将数据存储至数据库;同时,后端对外提供统一的数据接口,支持测试结果查询、跨版本对比以及性能趋势分析等功能,为前端的可视化展示提供服务支撑。
	
	\item
	数据库模块作为测试结果的持久化存储中心,统一保存所有历史测试数据,支持按照芯片型号、Commit号、优化等级、测试集以及测试样例等多维条件进行快速检索。这一设计为性能回归分析、版本对比以及长期趋势观察提供了坚实的数据基础。
	
	\item
	前端模块则为用户提供直观、交互友好的可视化界面。用户可以通过前端查看单个提交版本的测试结果,对比不同提交之间的性能差异,或以时间序列方式观察性能变化趋势等。同时,前端支持灵活的数据筛选操作,例如仅查看片上测试的性能指标,相关请求由后端实时处理并从数据库中返回结果进行展示。
	
\end{itemize}

\begin{figure}[htbp]
	\centering
	\resizebox{0.7\linewidth}{!}{
		\begin{tikzpicture}[
			box/.style={
				draw, rectangle, rounded corners, align=center,
				minimum width=5.4cm, minimum height=1.1cm
			},
			dashedbox/.style={
				draw, rectangle, dashed, rounded corners,
				inner sep=0.6cm
			},
			arr/.style={->, line width=0.4pt}
			]
			
			%==================== 数据入口 ====================
			\node[box] (perfrepo) {Perf 仓库\\(性能报告与评估结果)};
			\node[dashedbox, fit=(perfrepo), label=above:{数据入口}] {};
			
			%==================== 服务层(核心) ====================
			\node[box, right=2.0cm of perfrepo] (backend) {
				后端模块\\
				数据拉取与结构化解析\\
				统一数据服务接口
			};
			
			\node[dashedbox, fit=(backend),
			label=above:{服务层(Back-End)}] {};
			
			%==================== 展示层 ====================
			\node[box, above=2.2cm of backend] (frontend) {
				前端模块\\
				可交互性能展示与对比分析
			};
			
			\node[dashedbox, fit=(frontend),
			label=above:{展示层(Front-End)}] {};
			
			%==================== 数据层 ====================
			\node[box, below=2.2cm of backend] (db) {
				数据库模块\\
				历史数据持久化存储
			};
			
			\node[dashedbox, fit=(db),
			label=above:{数据层(Database)}] {};
			
			%==================== 连线(全部正交、无文字) ====================
			\draw[arr] (perfrepo.east) -- (backend.west);
			\draw[arr] (backend.north) -- (frontend.south);
			\draw[arr] (backend.south) -- (db.north);
			
		\end{tikzpicture}
	}
	\caption{评估子系统整体结构与三层架构设计图}
	\label{fig:eval-subsystem}
\end{figure}


%*********************************************************************
% 4.3 平台实现
%*********************************************************************
\section{平台实现}


% 4.3.1 面向DSP的YARPGen实现
\subsection{面向DSP的YARPGen实现}
YARPGen是通用的编译器测试用例生成工具,其默认设计面向通用CPU架构及具备标准运行时环境的C/C++程序。它假定目标平台具备标准库的支持,并可依赖stdio等通用接口完成程序结果的输出与验证。但DSP平台在硬件特性、调试机制以及运行环境方面均与通用处理器存在显著差异,这使得原生YARPGen难以直接应用于DSP编译器流水线的自动化验证。

\par

为了使随机测试生成机制能够适配到DSP编译器后端,本节对YARPGen进行了面向DSP架构的定制化扩展。整体修改遵循可复用和可扩展的设计原则,在保留YARPGen原有随机生成能力的基础上,新增了对DSP平台特有硬件约束与运行时环境的支持。具体改动体现在以下几个方面。


\subsubsection{1.引入DSP专用调试头文件}
DSP编译器的测试与验证依赖架构专用的调试与输出接口,而非标准的stdio输出机制。为保证生成的随机测试程序能够在DSP环境中正确输出运行结果,需在测试用例中引入特定头文件以支持调试输出和结果对比。在YARPGen的emitCheckFunc阶段,通过条件编译区分通用平台和DSP平台,为DSP环境添加专用调试头文件,并声明调试输出相关的函数接口。相关代码片段如下所示:

\lstset{language=c++}
\begin{lstlisting}
void emitCheckFunc(std::ostream &stream) {
	std::ostream &out_file = stream;
	out_file << "#ifdef DSP_VALIDATION\n";
	out_file << "#include <swift_debug.h>\n";
	out_file << "#else\n";
	out_file << "#include <stdio.h>\n\n";
	out_file << "#endif\n";
	...... }
\end{lstlisting}

其中,DSP\_VALIDATION用于标识当前测试是否运行于DSP验证环境。该宏由测试流水线在编译阶段统一注入,从而保证同一套测试生成逻辑可同时服务于通用平台与DSP平台,避免代码分叉维护。通过该方式,YARPGen在DSP模式下能够调用调试头文件中提供的调试输出接口,实现与DSP仿真器或硬件调试环境的对接。


\subsubsection{2.添加DSP专用调试输出逻辑}
在通用平台中,YARPGen通过标准输出对程序执行结果进行校验;而在DSP平台上,测试结果需要通过专用调试通道输出,以便测试框架进行采集与比对。为此,在YARPGen的emitMain阶段中,通过条件编译为DSP环境下添加调试函数调用,用于输出校验结果。


\subsubsection{3.引入目标架构选项与条件编译参数}
YARPGen的原生版本仅区分生成语言,而不会区分目标后端架构。为了支持DSP架构的相关逻辑,本文在YARPGen中引入了目标架构选项,并通过条件编译参数来控制测试生成与代码发射阶段的具体行为。这一改动为后续引入DSP架构的Intrinsic集提供了基础设施支持。


\subsubsection{4.添加DSP架构的Intrinsic}
为高效测试DSP编译器的行为,尤其是指令选择阶段对Intrinsic的处理,本文在YARPGen中引入了DSP架构的Intrinsic集,并将其纳入随机测试生成范围。由于YARPGen原生不支持Intrinsic,于是需要拓展其功能,在IRNode中引入新的Intrinsic选项,并根据操作数数量进行分类,如图\ref{fig:irnode_struct}所示。以SIN函数为例,对YARPGen的IRNode生成逻辑进行了扩展。

\begin{itemize}
	\item
	createHelper:核心作用是创建SIN函数调用表达式,具体包括生成算术表达式参数、处理布尔类型转换,最终返回封装后的IntrinsicExpr指针。该函数是YARPGen中表达式生成逻辑的核心组件,主要用于随机生成包含SIN函数调用的测试用例。
	
	\item
	propagateType:核心作用是对SIN函数的两个参数表达式进行类型推导、类型提升和类型统一,最终确定SIN函数调用表达式的返回值类型。
	
	\item 
	evaluate:在测试生成或运行时阶段计算SIN函数调用的结果,具体包括处理参数的类型转换、未定义行为判断,以及最终返回计算后的标量值。该函数是YARPGen中表达式求值逻辑的核心组件,主要用于确定SIN函数调用的实际结果值。
	
	\item
	emitCDefinitionImpl:核心作用是为C语言环境生成SIN函数的宏定义实现。因为C语言标准库中没有通用的SIN函数,该方法通过输出C语言的语句表达式宏,为SIN提供兼容C语言的实现,保证生成的测试用例在C环境下能正常编译运行。
	通过上述的定制化扩展,YARPGen被成功适配到DSP编译器的流水线测试场景中。基于条件编译的设计使得:同一套随机测试生成逻辑可同时服务于通用平台与DSP平台;DSP专用Intrinsic能够被系统性纳入随机测试覆盖范围。
	
\end{itemize}

\begin{figure}[htbp]
	\centering
	\includegraphics[width=0.8\textwidth]{pics/irnode_struct.png}
	\caption{拓展后的IRNode结构图}
	\label{fig:irnode_struct}
\end{figure}

对应的类层次结构如图\ref{fig:yarpgen_class}所示。

\begin{figure}[htbp]
	\centering
	\includegraphics[width=1\textwidth]{pics/yarpgen_class.png}
	\caption{扩展后的类结构}
	\label{fig:yarpgen_class}
\end{figure}


% 4.3.2 测试子系统实现
\subsection{测试子系统实现}
测试子系统在保留原有功能的基础上,主要实现了两方面的扩展:一是引入对编译器性能指标的自动化采集能力;二是集成面向DSP架构定制的YARPGen,并将完整的测试流程纳入CI。相应地,系统实现主要包括性能指标测量支持的引入以及测试脚本逻辑的重构两个方面。


\subsubsection{1.添加测量性能指标的支持}
在对VLIW架构DSP编译器进行性能评估时,通常关注以下几项核心指标:程序运行周期数、生成代码尺寸、汇编代码质量以及指令打包效率。这些指标分别从执行效率、存储占用、代码可读性以及指令级并行优化能力等维度,综合反映了编译器后端在目标平台上的性能表现。为准确获取上述不同维度的性能数据,系统采用了差异化的采集策略。

\begin{itemize}
	\item
	代码尺寸的测量通过DSP工具链中的llvm-size完成,工具通过对编译完成后的目标文件进行分析,来自动统计生成代码所占用的存储空间。
	
	\item
	汇编代码的获取则依赖于DSP工具链中的反汇编器,对编译完成后的目标文件进行反汇编,生成对应的汇编输出,为后续的指令级分析提供基础数据。
	
	\item 
	指令打包效率由DSP工具链的模拟器进行统计,模拟器能够在程序执行过程中收集指令发射和打包相关信息,从而反映编译器在指令调度与并行化方面的优化效果。
	
	\item
	运行周期数的获取在实现上具有一定复杂性。在模拟器环境中,程序运行周期数可直接由模拟器在程序执行过程中提供;而在实际DSP芯片上,则需依赖底层驱动库所暴露的硬件计数器接口进行测量。为将周期数测量无缝集成至自动化测试流水线,平台对原有驱动库进行了重构与精简,将其提取为独立、可重定向的模块,仅保留初始化与时间戳读取两个核心函数。在测试程序的链接阶段,该模块被统一链接至主函数中,从而实现对程序执行前后时间戳的自动采集与周期数计算。通过这一方式,芯片上的运行周期数测量能够与编译、执行流程紧密结合,实现流水线化和自动化。
	
\end{itemize}


\subsubsection{2.重构测试脚本并添加YARPGen}
原有的测试脚本逻辑仅关注测试样例的功能正确性验证,无法满足性能评估平台对多维性能数据采集的需求,为此,需要对测试脚本的整体逻辑进行重写和扩展。在新的脚本中,测试流程在完成编译与执行后,会自动触发性能数据采集模块,分别获取运行周期数、代码大小、汇编输出以及指令打包效率等指标,并将其统一整理为结构化的性能报告。

\par

性能报告以标准化Json格式生成,便于后续自动解析和存储。在报告生成完成后,测试脚本会将测试结果进行打包,并自动上传至Perf仓库,作为后续评估和可视化分析的数据源。

\par

此外为了提升测试覆盖度,CI脚本中还集成了YARPGen的随机样例生成逻辑,在每次测试触发时可自动生成一批新的随机测试程序,并与已有测试样例共同参与测试执行。通过这一机制,测试流水线不仅能够验证固定测试集上的功能和性能表现,还能够持续引入新的输入场景,提升对潜在缺陷的发现能力。


% 4.3.3 评估子系统实现
\subsection{评估子系统实现}
评估子系统用于对测试阶段产生的大量性能数据进行集中管理、分析与展示,其实现采用典型的三层架构,包括前端展示层、后端服务层以及数据库存储层。各层在功能上相互解耦,在数据流转上通过标准化接口进行协作,从而保证系统的可扩展性与维护性。


\subsubsection{1.数据展示层的实现}
数据展示层(下文简称前端)面向编译器性能分析与回归验证的实际需求,用于解决性能数据可视化难、跨版本对比成本高的问题。前端部分基于Vue框架实现交互界面,并结合Apache ECharts实现性能数据的图形化展示。通过前端界面,用户可以直观地查看单个提交版本的测试结果,对比不同提交之间的性能差异,并以时间序列形式观察性能指标的变化趋势。可视化展示方式有效降低了性能分析门槛,使开发者能够快速定位性能回退或异常波动问题。在实现过程中,前端通过组件化方式对不同功能模块进行拆分,具体模块如下:

\begin{itemize}
	\item 
	Dashboard:从宏观上展示DSP芯片与其他芯片在典型测试集上的性能对比结果,用于快速评估工具链性能水平。
	
	\item
	Summary:基于完整测试集统计不同提交版本下的平均代码大小与执行周期,用于观察性能随时间的整体演进趋势。
	
	\item
	History:针对单个测试样例,以时间序列方式展示其在不同提交下的性能变化,用于精确定位性能回退首次引入的位置。
	
	\item
	Compare:对任意两个提交版本进行逐样例性能对比,并支持汇编代码对照分析,用于解释性能差异的底层原因。
	
	\item
	Custom:支持上传本地测试结果并与主分支数据进行对比,便于在本地优化验证阶段提前发现潜在性能问题。
	
\end{itemize}

前端通过调用后端提供的REST接口获取性能数据,并在本地对数据进行初步处理和筛选,以减少不必要的数据传输,前端部分核心API如表\ref{tab:frontend_api_simple}所示。性能数据的可视化展示采用Apache ECharts实现,针对不同分析场景选用合适的图表类型:例如使用柱状图展示与其他芯片在相同测试集下执行周期数或代码大小上的对比结果,使用折线图展示执行周期数及代码大小随提交版本变化的趋势情况。通过使用这种方式,前端能够直观反映性能变化特征,帮助开发者快速识别性能回退或异常波动。

\begin{table}[htbp]
	\centering
	\caption{前端部分核心API列表}
	\label{tab:frontend_api_simple}
	
	\renewcommand{\arraystretch}{1.2}
	{\footnotesize
		\begin{tabularx}{\linewidth}{
				>{\raggedright\arraybackslash}m{2.5cm}
				>{\centering\arraybackslash}X
				>{\centering\arraybackslash}X
			}
			\toprule
			\textbf{接口名称} & \textbf{请求方式} & \textbf{功能描述} \\
			\midrule
			getOptLevelOptions & GET & 获取编译优化级别选项 \\
			getChipOptions & GET & 获取目标芯片选项列表 \\
			getSuiteOptions & GET & 获取测试集选项列表 \\
			getCaseOptions & GET & 获取测试用例名称选项列表 \\
			getAllCommits & GET & 获取所有代码提交版本信息 \\
			getCommitByHash & GET & 根据哈希值查询单个提交的详细信息 \\
			getHistoricalStatisticalInfo & POST & 查询指定条件下测试集的历史统计信息 \\
			getLatestPerfOfSuite & POST & 查询指定条件下测试集的最新性能数据 \\
			getHistoricalPerf & POST & 查询单个测试用例的历史性能趋势数据 \\
			getTwoCommitsComparison & POST & 对比两个提交版本的测试用例性能差异 \\
			getAssemblyComparison & POST & 对比两个提交版本的汇编代码差异 \\
			uploadLocalReport & POST & 上传本地性能测试报告并解析 \\
			removeAllCustomReports & GET & 清空所有已上传的本地自定义测试报告 \\
			\bottomrule
		\end{tabularx}
	}
\end{table}


\subsubsection{2.业务逻辑层的实现}
业务逻辑层(下文简称后端)是评估子系统的核心控制单元,负责性能数据的自动获取、解析建模以及对外接口服务。后端部分基于Spring Boot框架实现,负责性能数据解析、业务逻辑处理以及接口服务等任务。后端在启动或触发定时任务时,会从Perf仓库中拉取测试结果,并按照预定义的数据格式对原始JSON数据进行解析。解析过程中,后端会将测试环境信息(如目标处理器、优化等级、测试平台)与单个测试样例的性能指标进行关联建模,从而形成结构化的数据对象。这种建模方式使得不同维度的性能数据能够被统一管理,为后续的多条件查询和对比分析提供基础。

\par

随后,这些数据通过MyBatis持久层框架映射到数据库表结构中,实现性能数据的持久化存储。为保证在多用户查询或批量数据写入场景下系统的稳定性,后端引入Druid数据库连接池,对数据库连接进行统一管理与复用,有效降低了连接创建和释放的开销。

\par

在接口设计方面,后端向前端提供了一组面向性能分析任务的REST接口,包括按提交版本查询、跨版本对比以及历史趋势获取等功能。复杂的数据筛选与聚合逻辑均在后端完成,从而降低前端实现复杂度,提高系统整体的模块化程度。后端业务逻辑层的实际实现架构及核心组件的调用关系如图\ref{fig:backend_impl}所示。

\begin{figure}[htbp]
	\centering
	\includegraphics[width=0.95\textwidth]{pics/backend_impl.jpg}
	\caption{后端业务逻辑层实现架构图}
	\label{fig:backend_impl}
\end{figure}


\subsubsection{3.数据存储层的实现}
数据存储层的作用是对业务逻辑层处理后的性能报告进行持久化存储,为了实现更高效的查询,本文采用Mysql关系型数据库来作为核心存储引擎。同时考虑到性能测试报告会随着工具链的持续迭代不断积累,在设计数据库时,对常用查询维度建立索引,以提升历史数据查询和对比分析的效率。该数据存储机制为评估子系统提供了稳定、可扩展的数据基础,有效支撑了后续章节中针对全局指令选择性能影响的实验分析。以record表为例,表\ref{tab:record_struct}对record表的核心字段、数据类型、约束规则及索引设计进行了详细说明。

\begin{table}[htbp]
	\centering
	\caption{record表结构与索引设计}
	\label{tab:record_struct}
	
	\renewcommand{\arraystretch}{1.2}
	{\footnotesize
		\begin{tabularx}{\linewidth}{
				>{\raggedright\arraybackslash}m{1cm}
				>{\centering\arraybackslash}X
				>{\centering\arraybackslash}X
				>{\centering\arraybackslash}X
			}
			\toprule
			\textbf{字段名} & \textbf{数据类型} & \textbf{约束} & \textbf{说明} \\
			\midrule
			id & varchar(64) & NOT NULL, PK & 测试记录主键 \\
			case\_name & varchar(128) & NOT NULL & 测试样例名称 \\
			status & varchar(16) & NOT NULL & 测试执行状态 \\
			target\_cpu & varchar(32) & NOT NULL & 目标CPU架构 \\
			opt\_level & int & NOT NULL & 编译优化级别 \\
			platform & varchar(16) & NOT NULL & 测试平台 \\
			testsuite & varchar(64) & NOT NULL & 测试集名称 \\
			hash & varchar(64) & NOT NULL, FK & 关联commit表的哈希值 \\
			cycle\_num & int & NOT NULL & 测试用例执行周期数 \\
			code\_size & int & NOT NULL & 编译后代码大小 \\
			bundle\_x & int & NOT NULL & 指令打包情况 \\
			\midrule
			\multicolumn{4}{l}{\textbf{索引说明:}} \\
			\multicolumn{4}{l}{1. PRIMARY (id):主键索引,加速单条测试记录的查询与修改;} \\
			\multicolumn{4}{l}{2. fk\_record\_commit\_hash (hash):外键索引,加速提交与测试记录的关联查询;} \\
			\multicolumn{4}{l}{3. idx\_record\_for\_assembly\_fk (target\_cpu, opt\_level, testsuite, case\_name):复合索引,优化汇编关联查询;} \\
			\multicolumn{4}{l}{4. idx\_record\_for\_stats (target\_cpu, platform, opt\_level, testsuite, hash):复合索引,加速性能统计类查询。} \\
			\bottomrule
		\end{tabularx}
	}
\end{table}


%*********************************************************************
% 4.4 本章小结
%*********************************************************************
\section{本章小结}
本章分析了DSP编译器在引入GlobalISel及相关优化后,现有CI测试在输入覆盖与结果分析方面存在的不足。针对现存的不足,本章设计并实现了一套面向DSP工具链的测试评估平台,通过引入随机测试生成工具、完善性能数据自动采集,并结合前后端与数据库的三层架构,实现了测试结果的统一管理与可视化分析。最终,形成了一套覆盖测试生成、流水线执行、性能采集与对比分析的完整流程,为后续GlobalISel优化效果的定量评估与回归分析提供了可靠支撑。
