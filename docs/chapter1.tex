\chapter{绪论}


%*********************************************************************
% 1.1 研究背景与意义
%*********************************************************************
\section{研究背景与意义}


% 1.1.1 DSP架构简介
\subsection{DSP架构简介}
DSP(Digital Signal Processor,数字信号处理)是一种用于处理数字信号的微处理器,其原理是先通过采样、量化与编码的技术手段,将连续变化的模拟信号转化为离散的数字信号,之后再借助相应算法对所得数字信号完成滤波、变换及压缩等一系列处理\cite{proakis2007digital},DSP的处理效率与精度直接影响终端设备的性能上限。自20世纪中叶诞生以来,DSP已从早期单一功能的专用硬件电路,演进为具备可编程能力、高并行运算特性的专用处理器\cite{wang2021advancing}。近些年来,随着哈佛架构的引入\cite{song2023overview}、以及单指令多数据(SIMD)结构\cite{flynn2009some}与超长指令字(VLIW)技术\cite{fisher1998very, rau2011instruction}的结合,DSP实现了从标量处理到大规模并行处理的跨越式发展。

\par

随着5G/6G通信\cite{giordani2020toward}、边缘智能计算\cite{satyanarayanan2009case}等新兴应用的爆发式增长,对DSP的性能提出了更为严格的要求:一方面,数据吞吐量需求呈指数级提升,需要更宽的并行运算单元与更高效的指令调度机制;另一方面,终端设备的低功耗需求日益突出,要求DSP在提升性能的同时实现精细化的功耗管控。在这个背景下,国内外科研机构与企业纷纷加大高性能DSP的研发投入,但由于核心架构设计、指令集开发等关键技术存在一定的技术壁垒,国内高性能DSP领域仍面临自主化程度不足的挑战,亟需具备完全自主知识产权的高性能DSP芯片及配套技术体系作为支撑。

\par

本文所探讨的是一款由本实验室自主研发的高性能DSP芯片(在下文中简称为DSP芯片)。DSP芯片基于哈佛架构设计,具备自主知识产权的指令集,通过四路并行向量处理(VP)单元、8槽位VLIW指令发射机制及九级流水线设计,构建了高并行、高实时的运算架构;在寄存器的设计上,DSP采用32个32位通用寄存器GR、640位宽向量寄存器VR以及专用循环访存寄存器组MOB等特殊寄存器的分层架构,能够适配标量与向量运算的需求。


% 1.1.2 编译器研究背景
\subsection{编译器研究背景}
编译器是一个将高级语言编写的程序转换成能在一台计算机上执行的等价目标代码或机器语言程序的软件系统\cite{muchnick1997advanced}。早期的计算机软件都是用汇编语言直接编写的,当人们发现为不同类型的处理器编写可重用软件的开销要明显高于编写编译器时,高级编程语言应运而生。20世纪50年代末期,与机器无关的编程语言被首次提出。随后,人们开发了几种实验性质的编译器。第一个编译器是由Grace Hopper于1952年为A-0系统编写的。但是1957年由John Backus领导的FORTRAN\cite{backus1957fortran}则是第一个被实现出具备完整功能的编译器。1960年,COBOL\cite{conway1963design}成为一种较早的能在多种架构下被编译的语言。此时的编译器无标准化编译流程,每个编译器均为特定语言与硬件定制。

\par

1964年IBM推出的PL/I语言编译器,首次实现“同时适配科学计算与商用数据处理”的多场景支持,验证了编译器的通用性潜力。1972年阿霍与乌尔曼提出的“词法分析-语法分析-语义分析-代码优化-代码生成”五阶段流程\cite{aho1972theory},如图\ref{fig:compile_flowchart}所示,成为全球编译器研发的标准框架;1975年UNIX系统与C语言的结合,催生了可移植编译器的研发——1978年贝尔实验室推出的C编译器\cite{johnson1978unix},通过前后端分离的设计,首次实现同一前端解析C语言,不同后端适配不同硬件,为编译器的跨架构适配提供了范式。

\begin{figure}[htbp]
	\centering
	\includegraphics[width=0.5\textwidth]{pics/compile_flowchart.jpg}
	\caption{编译器五阶段流程图}
	\label{fig:compile_flowchart}
\end{figure}

1987年理查德·斯托曼发起GNU项目\cite{stallman1988gnu},推出GCC编译器,通过开源模式吸引全球开发者参与,逐步支持C、C++、Fortran、Java等多语言,适配x86、ARM等主流架构,成为开源生态的核心基础设施;LLVM\cite{lattner2004llvm}由Chris于2000年在伊利诺伊大学创建,其模块化、可重定向的设计理念,为后续专用架构编译奠定了基础。
随着GPU、DSP、AI加速器等异构架构的兴起\cite{刘颖2014异构并行编程模型研究与进展},以及高性能计算、移动互联网等场景的需求,编译器的核心目标从通用适配转向架构专属优化,追求极致性能+低功耗的平衡。2003年NVIDIA推出CUDA\cite{guide2013cuda}平台及配套编译器,首次实现C语言到GPU指令的编译优化,通过单指令多线程(SIMT)指令映射策略\cite{nickolls2008scalable},充分挖掘GPU的并行计算潜力;2009年LLVM 2.6版本发布,其模块化架构被广泛用于异构芯片编译,如苹果的Clang编译器、ARM的商用编译器等,均通过定制LLVM后端实现架构专属优化。


% 1.1.3 指令选择研究背景
\subsection{指令选择研究背景}
指令选择是编译器后端的核心环节,其核心任务是将编译器的中间表示IR转换为目标硬件支持的机器指令,其决策质量直接决定了生成代码的执行效率、资源占用及硬件适配性\cite{bezbaruah2024comparative},是衔接编译前端语义表达与后端硬件执行的关键桥梁。从技术发展的视角来看,指令选择的发展始终围绕硬件架构复杂度提升与应用场景性能需求升级的双重驱动展开,形成了从局部适配到全局优化、从可执行性优先到能效比最优的迭代路径。

\par

早期编译器处于基础架构搭建阶段,以支持基础编程语言和简单硬件架构为目标,核心需求是从中间表示IR映射到目标机器指令,确保代码可执行性,对全局优化和编译效率要求较低。在这样的需求背景下,指令选择技术以局部指令选择为核心,形成了宏拓展机制、树覆盖算法\cite{hjort2016tree}以及有向无环图(Directed Acyclic Graph,DAG)\cite{smotherman1991efficient}这三类主流实现方案。其中,DAG匹配成为主流,通过将基本块内的指令表示为DAG,寻找最优指令组合。DAG匹配的核心思想是构建一个有向无环图,通过预定义的映射规则,实现IR到机器指令的精准匹配与局部优化。这种DAG图的优势在于能天然体现指令间的数据依赖关系,便于实现如公共子表达式消除这样的局部优化,不过其局限性也较为明显:DAG图的构建与优化依赖局部子图分析,难以感知函数级的全局依赖关系,在VLIW、宽位向量DSP等复杂架构中,易因局部最优决策导致全局资源浪费。

\par

随着多核心、向量扩展(如SIMD、SVE\cite{smotherman1991efficient})、异构计算(GPU、加速器)等架构普及,传统局部指令选择难以适配复杂硬件的指令集特性;高性能计算、嵌入式系统等场景对代码执行效率要求严苛,局部优化的性能天花板逐渐显现,亟需全局视角的指令选择策略;大型项目编译周期长,DAG等传统方法的编译开销成为瓶颈,需要更高效的指令映射方案。

\par

LLVM作为开源编译系统的主流选择,其生态需要一套能够统一多架构适配、兼顾编译速度与代码质量的指令选择框架。2019年的LLVM全球开发者大会将GlobalISel列为核心主题,正式推动其成为传统方案的替代者。相比于SelectionDAGISel,GlobalISel解决了全局优化问题:以整个函数为操作粒度进行操作,保留完整的IR信息,拥有更大的视野,支持跨基本块的指令优化(如全局寄存器分配、跨块指令合并);降低了编译开销:去除DAG中间表示,直接将IR转换为通用机器指令,简化指令映射流程;提升了模块化与复用性:采用可配置的流水线架构,不同目标架构可复用基础流程,仅需定制架构相关规则,符合模块化的思想。除此之外,SelectionDAGISel和FastISel截然不同,可以共享的代码非常少,而GlobalISel的构建方式使其能够实现代码的重用。

\par

近年来,LLVM社区持续推进GlobalISel的功能完善,针对AArch64、RISC-V、PowerPC、AMDGPU等主流架构完成适配,苹果、谷歌等企业也参与核心开发;从基础指令映射向深度优化演进,新增基于代价模型的指令合并、全局寄存器组选择、指令局部性优化等功能,缩小与传统方案的代码质量差距。


% 1.1.4 研究意义
\subsection{研究意义}
随着数字化浪潮的持续推进,嵌入式技术已渗透到工业控制、智能终端、物联网等众多领域,尤其在数字信号处理应用场景中,DSP处理器凭借其针对实时信号运算的专用架构设计、高并行处理能力及低功耗优势,成为自动驾驶、智能传感、通信基站等对响应速度与稳定性要求严苛的系统核心支撑。值得注意的是,即便当下DSP处理器的硬件算力已实现大幅提升,如何充分释放硬件潜能、通过编译优化与代码适配实现程序执行效率的精准提升,仍是尚未完全解决的关键课题。高性能硬件架构的潜力发挥取决于编译器的支撑能力,如何针对自研DSP的独特架构特性,设计高效的编译优化策略,提升代码执行效率并缩减代码尺寸,成为当前亟需解决的关键问题。

\par

本研究聚焦于编译器的指令选择阶段,针对自研DSP架构的专用化设计需求与实时信号处理场景的高性能诉求,开展了系统性的技术研发与优化工作。首先,突破传统局部指令选择的局限,基于GlobalISel框架实现了函数级全局指令选择功能,解决了传统DAG-based方案跨基本块优化能力缺失、编译开销大等痛点,为充分挖掘硬件全局优化潜力奠定基础;其次,针对自研DSP架构的核心特性,定制开发了指令合并优化,在提升编译速度的同时,显著优化了生成代码的执行效率与资源利用率;最后,构建了涵盖运行时间、代码尺寸以及打包情况等多维度指标的一站式测试评估平台,实现了对优化效果的精准量化与迭代验证。该研究不仅为自研DSP架构提供了高效、适配性强的编译支撑,填补了专用架构编译优化工具的空白,还通过全局指令选择与架构特性的深度协同,有效释放了DSP硬件算力,为实时信号处理、嵌入式等领域的高性能应用开发提供了技术保障,具有重要的工程实践价值与技术推广意义。


%*********************************************************************
% 1.2 国内外研究现状
%*********************************************************************
\section{国内外研究现状}


% 1.2.1 编译器研究现状
\subsection{编译器研究现状}
编译器作为连接软件与硬件的核心枢纽,用于将源代码翻译成机器可以执行的目标代码。从技术演进脉络来看,编译器架构已从早期封闭的定制化设计,逐步转向模块化、可扩展的开源框架,其中LLVM与GCC成为当前全球编译器研发的两大核心基石。

\par

GCC是第一个开源编译器,打破了传统的编译器技术格局,自1987年诞生以来,历经三十余年迭代,从单一C语言编译器成长为支持多语言、多架构的跨平台编译工具,深刻影响了UNIX-like系统、Linux发行版及嵌入式领域的发展\cite{griffith2002gcc}。GCC沿用了编译器领域经典的前端—优化—后端三段式架构设计\cite{johnson1978portable},在长期迭代中整合了常量传播、死代码消除、循环展开等一系列成熟的通用编译优化技术,能够满足多数场景下的基础编译需求。然而,受限于早期架构设计的历史惯性,该架构存在难以规避的固有缺陷:一方面,核心代码模块耦合度较高,编译流程的逻辑封装不够清晰,导致代码可读性欠佳,新开发者介入二次开发或功能调试时需花费大量时间梳理逻辑关联;另一方面,架构的拓展性不足,新增语言前端、适配新硬件指令集或集成定制化优化策略时,往往需要对现有核心代码进行大幅修改,开发成本高且易引入兼容性问题;此外,不同体系架构的适配逻辑缺乏统一的抽象层支撑,导致跨架构编译时的兼容性表现不佳,难以快速适配嵌入式、异构计算等场景下的小众架构或定制化芯片,这一问题在硬件架构日趋多元化的当下尤为突出。

\par

2000年LLVM的提出影响了编译器的发展轨迹,其突破传统编译器的紧耦合三段式架构,将编译流程拆解为独立的功能模块,各模块通过标准化接口通信,实现高度解耦,如图\ref{fig:llvm_architecture_diagram}所示。采用与目标平台、源语言无关的中间表示IR,作为连接前端与后端的桥梁。IR基于静态单赋值(SSA)\cite{bilardi1999static}形式设计,兼具高层语言的抽象性与底层指令的精准性,能够完整保留代码的语义信息与优化潜力。除此之外,LLVM还将编译优化逻辑封装为独立的Pass单元,形成可配置、可组合的优化流水线\cite{leidel2021toward}。

\begin{figure}[htbp]
	\centering
	\includegraphics[width=1\textwidth]{pics/llvm_architecture_diagram.png}
	\caption{LLVM架构图}
	\label{fig:llvm_architecture_diagram}
\end{figure}

依托模块化的整体架构设计、跨平台统一的中间表示以及可组合的优化机制,LLVM已发展成为跨平台编译器实现与高性能代码生成领域的重要基础设施。其良好的可扩展性与架构适配能力,使其被多种主流及新兴编程语言的编译器所采用,例如Swift、TinyGo和Rust等均在实现层面不同程度地依赖LLVM后端。除此之外,在国际上对LLVM的相关研究还有许多,如Joseph N. Huber\cite{huber2021case}等人针对ARM A64FX处理器(面向高性能计算的可扩展向量架构),提出基于LLVM工具链的SIMD代码优化方法论;S Fang等人\cite{fang2025retrofitting}针对LLVM、GCC等主流编译器自动向量化能力不足的问题,提出一套融合专用IR扩展+灵活向量化流水线的优化方案;Kavon Farvardin等人\cite{farvardin2020new}对SML/NJ编译器\cite{appel1991standard}开发了LLVM后端,解决了SML的模块系统、多态类型与IR的语义映射问题;Paschalis Mpeis等人\cite{mpeis2021developer}为Android ART开发了LLVM后端,解决了默认后端因优化粒度有限,难以满足交互式应用的性能需求问题。验证了 LLVM 在移动虚拟机编译器中的优化潜力;Lammich等人\cite{lammich2022refinement}提出了一种基于LLVM的逐步细化方法,将并行算法的验证推进至LLVM代码级别,通过结合LLVM IR和代码生成步骤,提高程序的性能;Walter Rocchia等人\cite{balasubramanian2024designing}提出利用LLVM编译器后端为RISC-V处理器设计新的AI指令集扩展,以提升边缘设备上神经网络的执行效率。Juan Carlos等人\cite{de2025source}提出一种基于遗传算法和LLVM优化的源代码混淆技术,旨在解决云计算等开放执行环境下的软件知识产权保护与反逆向工程问题;John Regehr等人\cite{fan2024high}提出一款高吞吐量、形式化方法辅助的LLVM专用变异模糊测试工具,旨在高效发现LLVM编译器优化阶段的潜在缺陷。

\par

随着国产芯片产业的快速发展,LLVM由于其开源、模块化程度高等特点,逐渐被引入到多种自主研发硬件平台的工具链建设中。围绕LLVM的相关研究与工程实践,国内学界和工业界主要关注如何结合国产处理器架构特点进行适配与优化,并在此基础上提升编译器对底层硬件算力的利用效率。刘玉等人\cite{刘玉2020基于}针对魂芯数字信号处理器的特殊架构,基于LLVM构建优化编译器,解决前代编译器代码质量低、硬件特性支持不足的问题。沈莉等人\cite{沈莉2023swllvm}针对我国自主研制的神威新一代超级计算机解决异构系统可编程性差、硬件算力释放不足的问题,基于LLVM定制开发优化编译器;吴忧\cite{吴忧2024}提出了基于启发式搜索的优化序列选择算法。


% 1.2.2 指令选择研究现状
\subsection{指令选择研究现状}
在LLVM编译器生态中,有三种指令选择的实现方式:面向快速编译的快速指令选择(FastISel)、基于DAG图的指令选择(SelectionDAGISel)以及全局指令选择(GlobalISel)。表\ref{tab:isel_compare}给出了三者在设计目标、IR形态与作用域等方面的差异。

\begin{table}[htbp]
	\centering
	\caption{LLVM不同指令选择框架对比}
	\label{tab:isel_compare}
	
	\renewcommand{\arraystretch}{1.2}
	{\footnotesize
		\begin{tabularx}{\linewidth}{
				>{\raggedright\arraybackslash}m{2.6cm}
				>{\centering\arraybackslash}m{3.6cm}
				>{\centering\arraybackslash}X
				>{\centering\arraybackslash}X
				>{\centering\arraybackslash}X
				>{\centering\arraybackslash}X
			}
			\toprule
			方案 & 设计目标 & 中间表示 & 作用域 & 编译速度 & 代码质量 \\
			\midrule
			FastISel & 最大化编译速度 & 无 & 基本块 & 快 & 低 \\
			SelectionDAGISel & 优化代码质量 & DAG & 基本块 & 慢 & 高 \\
			GlobalISel & 平衡编译速度和代码质量 & GMIR & 函数 & 中 & 中 \\
			\bottomrule
		\end{tabularx}
	}
\end{table}

\par

FastISel通常在O0优化等级下启用,主要面向的是需要快速编译的场景,其核心宗旨是通过牺牲生成代码的质量来提升编译速度。与传统指令选择依赖多阶段中间表示处理不同,FastISel在遍历LLVM IR过程中直接对指令进行处理。FastISel通过对IR表达式结构进行递归访问,并结合目标架构中预定义的简化指令模式,完成指令的即时生成,从而避免了DAG构建及全局分析等开销较大的步骤。

\par

SelectionDAGISel是LLVM中应用最广泛的指令选择技术,自LLVM早期版本起就一直作为框架的核心实现。在x86、ARM以及RISCV等主流架构中,尤其是在中高优化等级下仍是默认的指令选择方案。SelectionDAGISel的核心思想是将中间表示转换为SelectionDAG,并在有向无环图上进行模式匹配,以克服传统树匹配方法在表达共享子表达式方面的局限性。从研究角度上来看,这种以SSA语义为基础、在图结构上完成匹配的模型被认为是对传统局部树结构指令选择的重要补充和发展。Ebner等人\cite{Ebner2021}对指令选择中的匹配模型进行了理论扩展,提出了一种基于SSA的DAG匹配方法,将匹配过程从局部的树结构提升为全局的DAG结构,不仅增强了模式匹配的表达能力,也提高了将机器无关中间表示映射为机器相关指令时的效率与准确性。其更早的工作也曾将指令选择形式化为SSA图上的图文法解析和组合优化问题,探索全函数范围上的更优覆盖与代价最小化\cite{ebner2008generalized}。这些研究为更大范围、更强约束的指令选择提供了理论基础,但在工程上也带来求解复杂度与编译时间的挑战。

\par

GlobalISel最初在LLVM开发者会议上被提出,设计初衷是用于解决SelectionDAG在编译性能、可扩展性与复用性方面的痛点:SelectionDAG引入了专用中间表示并伴随较高的构建和合法化开销,而不同后端在DAG层的大量定制也导致维护成本上升\cite{Colombet2015GlobalISel}。GlobalISel作为一种现代指令选择的替代方案已受到广泛的关注,同时它也是AArch64架构下O0优化等级的默认选择器。

\par

除此之外,随着指令选择规则规模与复杂度增长,研究工作开始关注后端选择器相关的测试覆盖与缺陷发现,例如针对GlobalISel匹配表/选择路径的专门化测试与模糊测试。这些方向共同推动了指令选择从局部模式向可扩展、可验证、面向多目标优化的方向演进\cite{rong2024irfuzzer}。


2015年,苹果公司的Quentin Colombet提出:现有的指令选择框架SelectionDAGISel存在若干根本性局限,包括但不限于:编译速度缓慢、仅支持基本块级别的局部作用域、架构设计过于单一化等。多年来,开发者们投入了大量精力通过增加Target hooks和优化Pass来规避这些局限,但这些方案本身也带来了新的问题(如启发式算法精度不足、需提前预测指令选择器的执行行为等)与局限。他认为,当前已具备推出新一代指令选择框架GlobalISel的条件,该框架将从根源上解决上述问题,同时为提升代码生成质量创造新的可能\cite{Colombet2015GlobalISel}。次年,Quentin Colombet等人分享了GlobalISel在设计与实现层面取得的阶段性成果,同时明确后续仍需重点研发的技术方向\cite{Colombet2016GlobalISelDesign}。

\par

2017年3月,苹果公司的Justin Bogner指出其团队在指令选择测试中应用模糊测试与输入生成技术的实验过程及核心成果。深入探讨了在寻找高价值测试输入过程中的技术权衡,以及验证生成代码有效性的核心方案\cite{Bogner2017ISelFuzzing}。

\par

2019年,Daniel Sanders指出目前GlobalISel的开发重心主要集中在对各类目标架构的基础支持上,优化相关的工作投入相对有限\cite{Sanders2019GlobalISelCombiner}。近期,研发方向已转向优化能力的提升,目标是使其优化水平达到能够全面替代SelectionDAGISel的程度。并详细讲解Combiner的整体设计架构、支撑其运行的核心模块、它与 GlobalISel 其他组件的协同运作方式,以及该组件的测试和调试方法。

\par

2021年,Huang Zhufeng等人提出基于代价模型的指令合并优化以及全局寄存器组选择优化以及指令局部性优化等\cite{zhufeng2021optimization},有效解决了传统指令选择的全局优化能力缺失、编译开销大等问题,在申威平台实现了编译速度与代码质量的平衡。

\par

2022年,Kai Nacke等人分享了其为PowerPC目标架构实现GlobalISel框架的初步实践经验\cite{Nacke2022PowerPCGlobalISel},详细阐述了在适配过程中面临的架构特性适配挑战、与传统SelectionDAGISel后端的功能衔接方案,以及达成的阶段性成果(如核心指令集的无回退匹配、基础基准测试的编译通过率提升等),为其他架构的 GlobalISel 适配工作提供可复用的实践参考。

\par

2024年,有关GlobalISel的研究骤然增长,其中Pierre Houtryve提出为GlobalISel组合器基础设施新增输入/输出MIR模式支持\cite{Houtryve2024MIRPattern},该功能包含类PatFrag系统与类型推断机制,使开发者能够直接在TableGen中编写大量组合器规则;Tobias Stadler提出不再生成IR,而是直接输出GMIR,从而跳过代码生成流水线的首个转换阶段\cite{Stadler2024DirectGMIR}。在作者的应用场景中,该方案使编译速度提升了约 20\%;Jiahan Xie提出面向可扩展向量的突破性GlobalISel的实现方案\cite{Xie2024ScalableVectorGlobalISel},该方案以RISC-V向量扩展为目标场景。演讲深入探讨了支持可扩展向量算术逻辑单元及加载/存储指令过程中面临的核心挑战与创新解决方案;Madhur Amilkanthwar提出针对GlobalISel对部分指令和模式的支持不完善,导致其在该平台上需回退至传统的SelectionDAGISel的问题\cite{Amilkanthwar2024AArch64GlobalISel}。做出的核心贡献如下:通过在GlobalISel各编译阶段引入补丁,消除了其回退现象;同时对GlobalISel生成的代码进行了优化,显著缩小了其与SelectionDAGISel在AArch64高级SIMD平台上的性能差距。这些进展标志着GlobalISel框架的优化迈出了重要一步,使其向成为默认指令选择器的目标更近了一步。

\par

2025年,Nvidia的Neil Hickey提出:GlobalISel在所有后端的推广应用受到其指令选择场景覆盖不完全的限制,这导致其在部分场景下仍需回退至SelectionDAGISel\cite{Hickey2025GlobalISelCI}。为解决并监控这些局限性,Neil等人开发了一套专用的持续集成系统。该系统每日会自动构建最新版本的LLVM,通过编译一系列广泛的基准测试程序并记录回退事件,来为LLVM社区定位问题来源。


%*********************************************************************
% 1.3 论文研究内容和创新点
%*********************************************************************
\section{论文研究内容和创新点}
结合前文分析可以看出,传统的SelectionDAGISel在实践中暴露出多方面的局限性。一方面,其实现依赖规模较大的代码体系,随着功能扩展不断膨胀,增加了开发与调试成本;另一方面,受限于SelectionDAG/SDNode的数据结构设计,指令选择过程需要频繁的进行内存分配与释放,不仅限制了优化策略的表达空间,也在一定程度上增加了编译时间开销。

\par

作为LLVM官方推崇的替代方案,GlobalISel在架构设计上具有更好的模块化与可扩展性,已被AArch64、AMDGPU、RISC-V以及X86等主流后端采用。然而,在大部分平台上,其当前生成代码质量仍与成熟的SelectionDAGISel存在差距,尤其是在目标相关指令匹配与指令组合方面,GlobalISel仍有较大的优化空间。

\par

现有CI测试流程的性能评估缺陷:DSP工具链已具备Gitlab托管与提交后CI测试机制,在测试通过且负责人审核通过后方可合并代码,但现有测试仅反馈测试样例通过或未通过的结果,存在在很多不足:缺乏 Cycles(执行周期)、Code Size(代码体积)等关键性能统计信息;无法支持不同Commit版本的性能对比分析;大量含Bug或测试性提交的无效结果混杂,不便于有效结果检索与定位。为解决上述问题,本文聚焦以下三方面研究:

\begin{enumerate}
	\item
	针对DSP硬件指令集以及寄存器布局等硬件特性,完成全局指令选择框架的定制化实现。通过扩展TableGen中的指令描述模板,补充DSP架构相关的指令语义与操作码映射关系,并结合DSP的硬件特点,设计了相应的寄存器组划分规则以及类型合法化逻辑,解决了通用GlobalISel在嵌入式DSP平台上适配性不足的问题。在此基础上,构建了适用于DSP后端的指令选择基础框架,替代原有的SelectionDAGISel,为后续指令选择优化与功能扩展提供了更具可维护性的实现基础。
	
	\item
	针对DSP硬件特性(如硬件指令集、寄存器布局等),对全局指令选择机制进行针对性优化,提升生成代码质量。
	
	\item 
	设计并实现与Gitlab CI深度结合的性能测试评估平台:在代码合并后,系统会自动调用脚本来启动模拟器和芯片端上的测试任务,在测试结束后将测试数据传到Perf仓库;后端服务通过设置定时任务来拉取Perf仓库的数据并更新数据库;前端则负责用图表展示性能数据。这套方案不但支持多版本之间的性能对标,还能够清晰地捕捉到每一次提交导致的性能提升或退步,实现了优化效果的精准量化。
	
\end{enumerate}



%*********************************************************************
% 1.4 论文组织结构
%*********************************************************************
\section{论文组织结构}
本文一共分为六个章节,针对全局指令选择的实现与优化以及性能评估平台的实现来进行研究,论文结构如下:

\par

本文一共分为六个章节,围绕全局指令选择的实现与优化方法以及性能评估平台的设计与实现展开研究,论文整体结构安排如下。

\par

第一章为绪论部分,介绍了论文的研究背景与意义,重点分析了DSP架构与编译器技术的发展现状,梳理了指令选择技术,尤其是全局指令选择相关研究的国内外研究现状与存在的问题,在此基础上明确了本文的研究内容、技术路线和主要创新点。

\par

第二章为全局指令选择框架的实现部分。本章先对传统指令选择方案及其局限性进行了分析,并给出全局指令选择的基本概念和设计思路,之后分节详细阐述了全局指令选择的GMIR的生成、指令合法化、寄存器组选择以及机器指令选择这四个核心阶段的设计与实现。

\par

第三章为全局指令选择的优化策略的实现部分。本章先结合DSP架构特点从理论角度分析了全局指令选择优化的必要性和可行性,之后针对内存操作指令、乘法指令等典型应用场景,提出并实现了多种优化策略,提升了生成代码的执行效率和质量。

\par

第四章为测试评估平台的设计与实现部分。本章根据现有DSP编译器测试流程中存在的问题,确定了测试体系优化的需求与技术路径,之后从系统架构层面给出了两个子系统的整体设计方案,并在此基础上实现。

\par

第五章为实验和展示部分。实验部分从编译时间、代码尺寸和执行周期等多个维度来分别对全局指令选择实现的正确性及优化的有效性进行了验证,展示部分则通过运行截图来直观地展示测试评估平台的实现情况。

\par

第六章为总结与展望部分。本章对全文的研究工作进行了系统的总结,归纳了本文在全局指令选择实现、优化策略以及测试评估平台方面取得的主要成果,并对存在的不足与未来改进方向进行讨论。