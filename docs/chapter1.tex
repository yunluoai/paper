\chapter{绪论}


%*********************************************************************
% 1.1 研究背景与意义
%*********************************************************************
\section{研究背景与意义}


% 1.1.1 DSP架构简介
\subsection{DSP架构简介}
DSP(Digital Signal Processor,数字信号处理器)是一种用于处理数字信号的微处理器,其原理是通过采样、量化与编码的技术手段,将连续变化的模拟信号转化为离散的数字信号,之后再借助相应算法对所得数字信号完成滤波、变换及压缩等一系列处理\cite{proakis2007digital},DSP的处理效率与精度直接影响终端设备的性能上限。自20世纪中叶诞生以来,DSP已从早期单一功能的专用硬件电路,演进为具备可编程能力、高并行运算特性的专用处理器\cite{wang2021advancing}。近些年来,随着哈佛架构的引入\cite{song2023overview}、以及单指令多数据(SIMD)结构\cite{flynn2009some}与超长指令字(VLIW)技术\cite{fisher1998very, rau2011instruction}的结合,DSP实现了从标量处理到大规模并行处理的跨越式发展。

\par

随着5G/6G通信\cite{giordani2020toward}、边缘智能计算\cite{satyanarayanan2009case}等新兴应用的爆发式增长,DSP面临着前所未有的性能与能效双重挑战:一方面,数据吞吐量需求呈指数级提升,需要更宽的并行运算单元与更高效的指令调度机制;另一方面,终端设备的低功耗需求日益突出,要求DSP在提升性能的同时实现精细化的功耗管控。在这个背景下,国内外科研机构与企业纷纷加大高性能DSP的研发投入,但由于核心架构设计、指令集开发等关键技术存在一定的技术壁垒,国内高性能DSP领域仍存在自主化程度不足的问题,因此,构建具备完全自主知识产权的高性能DSP芯片及端到端技术体系,已成为当前亟待解决的重要课题。

\par

本文所探讨的是一款由本实验室自主研发的高性能DSP芯片(在下文中简称为DSP芯片)。DSP芯片基于哈佛架构设计,具备自主知识产权的指令集,通过四路并行向量处理(VP)单元、8槽位VLIW指令发射机制及九级流水线设计,构建了高并行、高实时的运算架构;在寄存器的设计上,DSP采用32个32位通用寄存器GR、640位宽向量寄存器VR以及专用循环访存寄存器组MOB等特殊寄存器的分层架构,能够适配标量与向量运算的需求。


% 1.1.2 编译器研究背景
\subsection{编译器研究背景}
编译器是一个将高级语言编写的程序转换成能在一台计算机上执行的等价目标代码或机器语言程序的软件系统\cite{muchnick1997advanced}。早期的计算机软件都是用汇编语言直接编写的,当人们发现为不同类型的处理器编写可重用软件的开销要明显高于编写编译器时,高级编程语言应运而生。20世纪50年代末期,与机器无关的编程语言被首次提出。随后,人们开发了几种实验性质的编译器。第一个编译器是由Grace Hopper于1952年为A-0系统编写的。1957年由John Backus领导的FORTRAN\cite{backus1957fortran}则是第一个具备完整功能的编译器。1960年,COBOL\cite{conway1963design}作为一种具备跨平台移植能力的高级编程语言出现,此时的编译器无标准化编译流程,每个编译器均为特定语言与硬件定制。

\par

1964年IBM公司推出的PL/I语言编译器,首次实现了科学计算与商用数据处理两大领域的一体化支持,打破了此前编译器针对单一应用场景设计的局限,验证了编译器的通用性潜力。1972年阿霍与乌尔曼提出的“词法分析-语法分析-语义分析-代码优化-代码生成”五阶段流程\cite{aho1972theory},如图\ref{fig:compile_flowchart}所示,成为全球编译器研发的标准框架;1975年UNIX系统与C语言的结合,催生了可移植编译器的研发——1978年贝尔实验室推出的C编译器\cite{johnson1978unix},通过前后端分离的设计,首次实现同一前端解析C语言,不同后端适配不同硬件,为编译器的跨架构适配提供了范式。

\begin{figure}[htbp]
	\centering
	\includegraphics[width=0.45\textwidth]{pics/compile_flowchart.jpg}
	\caption{编译器五阶段流程图}
	\label{fig:compile_flowchart}
\end{figure}

1987年理查德·斯托曼发起GNU项目\cite{stallman1988gnu},推出GCC编译器,通过开源模式吸引全球开发者参与,逐步支持C、C++、Fortran、Java等多语言,适配x86、ARM等主流架构,成为开源生态的核心基础设施;2000年Chris在伊利诺伊大学创建了LLVM\cite{lattner2004llvm}项目,其模块化、可重定向的设计理念为后续专用架构编译奠定了基础。

\par

随着GPU、DSP、AI加速器等异构架构的兴起\cite{刘颖2014异构并行编程模型研究与进展},编译器的核心目标从通用适配转向架构专属优化,追求极致性能+低功耗的平衡。2003年NVIDIA推出CUDA\cite{guide2013cuda}平台及配套编译器,首次实现C语言到GPU指令的编译优化,通过单指令多线程(SIMT)指令映射策略\cite{nickolls2008scalable},充分挖掘GPU的并行计算潜力;2009年LLVM 2.6版本发布,其模块化架构被广泛用于异构芯片编译。如苹果的Clang编译器、ARM的商用编译器等,均通过定制LLVM后端实现架构专属优化。


% 1.1.3 指令选择研究背景
\subsection{指令选择研究背景}
指令选择是编译器后端的核心环节,核心任务是将平台无关的中间表示转换为平台相关的机器指令,其决策质量直接决定了生成代码的执行效率、资源占用及硬件适配性\cite{bezbaruah2024comparative},是衔接编译器前端语义表达与后端硬件执行的关键桥梁。

\par

早期编译器的目标是支持基础编程语言与简单的硬件架构,其核心任务是将高级语言或中间表示映射为目标机器的可执行指令,以保证程序的正确运行。该阶段的编译器设计侧重于功能的实现与代码生成的基本正确性,对全局优化能力与编译过程本身的效率要求相对较低。在这样的需求背景下,指令选择技术以局部指令选择为核心,形成了宏拓展机制、树覆盖算法\cite{hjort2016tree}以及DAG(Directed Acyclic Graph,有向无环图)\cite{smotherman1991efficient}三类实现方案。其中,基于DAG的指令选择成为主流方法。该方法将基本块内的指令序列表示为DAG结构,通过预定义的树模式在图中进行子图匹配,以寻找最优的指令组合。这种DAG图的优势在于能天然体现指令间的数据依赖关系,便于实现如公共子表达式消除这样的局部优化,不过其局限性也较为明显:DAG图的构建与优化依赖局部子图分析,难以感知函数级的全局依赖关系,在VLIW、宽位向量DSP等复杂架构中,易因局部最优决策导致全局资源浪费。

\par

随着多核心处理器、向量化扩展(如SIMD、SVE\cite{smotherman1991efficient})以及异构计算单元(GPU、专用加速器)等复杂硬件架构的普及,传统基于局部基本块的指令选择方法难以充分适配其多样化的指令集特性。同时,在高性能计算与嵌入式系统等场景中,对代码执行效率的要求日益严苛,局部优化所能达到的性能上限已逐渐无法满足需求,亟需引入具备全局视角的指令选择策略。此外,大型软件项目的编译时间长,传统基于DAG等结构的指令选择方法因其较高的编译开销而成为效率瓶颈,推动了对更轻量、更高效的指令映射方案的需求。

\par

LLVM作为开源编译系统的主流选择,其生态需要一套能够统一多架构适配、兼顾编译速度与代码质量的指令选择框架。2019年的LLVM全球开发者大会将GlobalISel(全局指令选择)列为核心主题,正式推动其成为传统方案的替代者。相比于SelectionDAGISel(基于DAG图的指令选择),GlobalISel的优势如下:

\begin{itemize}
	\item
	更多的优化空间:以整个函数为操作粒度进行操作,保留完整的IR信息,拥有更大的视野,支持跨基本块的指令优化。
	
	\item
	更低的编译开销:去除DAG中间表示,直接将IR转换为通用机器指令,简化指令映射流程;
	
	\item 
	模块化与复用性:采用可配置的流水线架构,不同目标架构可复用基础流程,仅需定制架构相关规则,符合模块化的思想。
	
\end{itemize}

近年来,LLVM社区持续推进GlobalISel的功能完善与架构适配,针对X86、AArch64、RISC-V以及AMDGPU等主流架构已完成适配。框架已从基础的指令映射阶段演进至深度优化阶段,陆续引入了基于代价模型的指令合并、全局寄存器组选择策略以及指令局部性优化等关键功能,从而逐步缩小了与传统指令选择方案在生成代码质量上的差距。


% 1.1.4 研究意义
\subsection{研究意义}
在数字信号处理场景中,DSP处理器凭借强大的并行处理能力及优异的能效表现,已经成为自动驾驶、智能传感和通信基站等对实时性与可靠性要求极高系统的核心计算单元。尽管当前DSP硬件在计算能力方面已取得显著进展,但编译技术的瓶颈却制约着其性能的充分释放。如何针对自研DSP架构的特性,在编译器层面进行深度适配与优化,以实现程序执行效率的最大化及代码存储密度的提升,已成为当前亟待解决的核心技术问题。

\par

面向自研DSP架构的专用化设计需求以及实时信号处理场景的高性能目标,本文开展了一系列系统性研究与工程实现。首先,针对传统指令选择方案在跨基本块优化能力、编译可扩展性等方面存在的局限,本文设计并实现了面向DSP的GlobalISel框架。该框架以函数为粒度进行指令映射与优化,有效克服了传统基于DAG的方法在跨基本块优化能力不足、编译耗时较长等方面的局限,为充分发挥硬件并行与优化潜力提供了基础支持。其次,为充分发挥自研DSP架构的指令集优势,本文设计并实现了一系列针对性的指令合并与优化策略,在提升编译效率的同时,显著改善了生成代码的执行性能与资源使用效率。最后,为了解决现有测试体系存在的问题,本文构建了一套完整的测试评估平台。该平台集成随机测试生成、多维度性能采集和自动化分析报告等功能,能够系统性地对编译生成的代码在运行周期、存储占用、指令打包效率等关键指标上进行量化评估,为编译器的性能评估提供了可靠的数据支撑。

\par

本研究通过引入全局指令选择框架、实施针对性的架构定制优化以及建立系统化的评估体系,为自研DSP处理器构建了一套高效、稳定且可扩展的编译工具链,同时也为高实时性、高能效的嵌入式信号处理应用开发提供了坚实的技术基础,具有明确的工程应用价值与广泛的行业推广前景。


%*********************************************************************
% 1.2 国内外研究现状
%*********************************************************************
\section{国内外研究现状}


% 1.2.1 编译器研究现状
\subsection{编译器研究现状}
编译器作为连接软件与硬件的核心枢纽,用于将源代码翻译成机器可以执行的目标代码。从技术演进脉络来看,编译器架构已从早期封闭的定制化设计,逐步转向模块化、可扩展的开源框架,其中LLVM与GCC成为当前全球编译器研发的两大核心基石。

\par

GCC作为第一个开源编译器,打破了传统的编译器技术格局,自1987年诞生以来,历经三十余年迭代,从单一C语言编译器成长为支持多语言、多架构的跨平台编译工具,深刻影响了UNIX-like系统、Linux发行版及嵌入式领域的发展\cite{griffith2002gcc}。GCC沿用了编译器领域经典的前端—优化—后端三段式架构设计\cite{johnson1978portable},在长期迭代中整合了常量传播、死代码消除、循环展开等一系列成熟的通用编译优化技术,能够满足多数场景下的基础编译需求。然而,受限于早期架构设计的历史惯性,该架构存在难以规避的固有缺陷:一方面,核心代码模块耦合度较高,编译流程的逻辑封装不够清晰,导致代码可读性欠佳,新开发者介入二次开发或功能调试时需花费大量时间梳理逻辑关联;另一方面,架构的拓展性不足,新增语言前端、适配新硬件指令集或集成定制化优化策略时,往往需要对现有核心代码进行大幅修改,开发成本高且易引入兼容性问题;此外,不同体系架构的适配逻辑缺乏统一的抽象层支撑,导致跨架构编译时的兼容性表现不佳,难以快速适配嵌入式、异构计算等场景下的小众架构或定制化芯片,这一问题在硬件架构日趋多元化的当下尤为突出。

\par

2000年LLVM的提出影响了编译器的发展轨迹,LLVM突破了传统编译器的紧耦合三段式架构,将编译流程拆解为独立的功能模块,各模块通过标准化接口通信,实现高度解耦,其架构如图\ref{fig:llvm_architecture_diagram}所示。LLVM采用与目标平台、源语言无关的中间表示IR,作为连接前端与后端的桥梁。IR基于静态单赋值(SSA)\cite{bilardi1999static}形式设计,兼具高层语言的抽象性与底层指令的精准性,能够完整保留代码的语义信息与优化潜力。除此之外,LLVM还将编译优化逻辑封装为独立的Pass单元,形成可配置、可组合的优化流水线\cite{leidel2021toward}。

\begin{figure}[htbp]
	\centering
	\includegraphics[width=1\textwidth]{pics/llvm_architecture_diagram.png}
	\caption{LLVM架构图}
	\label{fig:llvm_architecture_diagram}
\end{figure}

依托模块化的整体架构设计、跨平台统一的中间表示以及可组合的优化机制,LLVM已发展成为跨平台编译器实现与高性能代码生成领域的重要基础设施。其良好的可扩展性与架构适配能力,使其被多种主流及新兴编程语言的编译器所采用,例如Swift、TinyGo和Rust等均在实现层面不同程度地依赖LLVM后端。除此之外,在国际上对LLVM的相关研究还有许多,如Joseph N. Huber\cite{huber2021case}等人针对ARM A64FX处理器(面向高性能计算的可扩展向量架构),提出基于LLVM工具链的SIMD代码优化方法论;S Fang等人\cite{fang2025retrofitting}针对LLVM、GCC等主流编译器自动向量化能力不足的问题,提出一套融合专用IR扩展+灵活向量化流水线的优化方案;Kavon Farvardin等人\cite{farvardin2020new}对SML/NJ编译器\cite{appel1991standard}开发了LLVM后端,解决了SML的模块系统、多态类型与IR的语义映射问题;Paschalis Mpeis等人\cite{mpeis2021developer}为Android ART开发了LLVM后端,解决了默认后端因优化粒度有限,难以满足交互式应用的性能需求问题。验证了 LLVM 在移动虚拟机编译器中的优化潜力;Lammich等人\cite{lammich2022refinement}提出了一种基于LLVM的逐步细化方法,将并行算法的验证推进至LLVM代码级别,通过结合LLVM IR和代码生成步骤,提高程序的性能;Walter Rocchia等人\cite{balasubramanian2024designing}提出利用LLVM编译器后端为RISC-V处理器设计新的AI指令集扩展,以提升边缘设备上神经网络的执行效率。Juan Carlos等人\cite{de2025source}提出一种基于遗传算法和LLVM优化的源代码混淆技术,旨在解决云计算等开放执行环境下的软件知识产权保护与反逆向工程问题;John Regehr等人\cite{fan2024high}提出一款高吞吐量、形式化方法辅助的LLVM专用变异模糊测试工具,旨在高效发现LLVM编译器优化阶段的潜在缺陷。

\par

随着国产芯片产业的快速发展,LLVM由于其开源、模块化程度高等特点,逐渐被引入到多种自主研发硬件平台的工具链建设中。围绕LLVM的相关研究与工程实践,国内学界和工业界主要关注如何结合国产处理器架构特点进行适配与优化,并在此基础上提升编译器对底层硬件算力的利用效率。刘玉等人\cite{刘玉2020基于}针对魂芯数字信号处理器的特殊架构,基于LLVM构建优化编译器,解决前代编译器代码质量低、硬件特性支持不足的问题。沈莉等人\cite{沈莉2023swllvm}针对我国自主研制的神威新一代超级计算机解决异构系统可编程性差、硬件算力释放不足的问题,基于LLVM定制开发优化编译器;吴忧\cite{吴忧2024}提出了基于启发式搜索的优化序列选择算法。


% 1.2.2 指令选择研究现状
\subsection{指令选择研究现状}
在LLVM编译器生态中,有三种指令选择的实现方式:面向快速编译的FastISel(快速指令选择)、基于DAG图的SelectionDAGISel以及GlobalISel。表\ref{tab:isel_compare}给出了三者在设计目标、IR形态与作用域等方面的差异。

\begin{table}[htbp]
	\centering
	\caption{LLVM不同指令选择框架对比}
	\label{tab:isel_compare}
	
	\renewcommand{\arraystretch}{1.2}
	{\footnotesize
		\begin{tabularx}{\linewidth}{
				>{\raggedright\arraybackslash}m{2.6cm}
				>{\centering\arraybackslash}m{3.6cm}
				>{\centering\arraybackslash}X
				>{\centering\arraybackslash}X
				>{\centering\arraybackslash}X
				>{\centering\arraybackslash}X
			}
			\toprule
			方案 & 设计目标 & 中间表示 & 作用域 & 编译速度 & 代码质量 \\
			\midrule
			FastISel & 最大化编译速度 & 无 & 基本块 & 快 & 低 \\
			SelectionDAGISel & 优化代码质量 & DAG & 基本块 & 慢 & 高 \\
			GlobalISel & 平衡编译速度和代码质量 & GMIR & 函数 & 中 & 中 \\
			\bottomrule
		\end{tabularx}
	}
\end{table}

\par

FastISel通常在O0优化等级下启用,主要面向的是需要快速编译的场景,其核心宗旨是通过牺牲生成代码的质量来提升编译速度。与传统指令选择依赖多阶段中间表示处理不同,FastISel在遍历LLVM IR过程中直接对指令进行处理。FastISel通过对IR表达式结构进行递归访问,并结合目标架构中预定义的简化指令模式,完成指令的即时生成,从而避免了DAG构建及全局分析等开销较大的步骤。

\par

SelectionDAGISel是LLVM中应用最广泛的指令选择技术,自LLVM早期版本起就一直作为框架的核心实现。在x86、ARM以及RISCV等主流架构中,尤其是在中高优化等级下仍是默认的指令选择方案。SelectionDAGISel的核心思想是将中间表示转换为SelectionDAG,并在有向无环图上进行模式匹配,以克服传统树匹配方法在表达共享子表达式方面的局限性。从研究角度上来看,这种以SSA语义为基础、在图结构上完成匹配的模型被认为是对传统局部树结构指令选择的重要补充和发展。Ebner等人\cite{Ebner2021}对指令选择中的匹配模型进行了理论扩展,提出了一种基于SSA的DAG匹配方法,将匹配过程从局部的树结构提升为全局的DAG结构,不仅增强了模式匹配的表达能力,也提高了将机器无关中间表示映射为机器相关指令时的效率与准确性。其更早的工作也曾将指令选择形式化为SSA图上的图文法解析和组合优化问题,探索全函数范围上的更优覆盖与代价最小化\cite{ebner2008generalized}。这些研究为更大范围、更强约束的指令选择提供了理论基础,但在工程上也带来求解复杂度与编译时间的挑战。

\par

GlobalISel最早是在LLVM开发者会议上被提出,其设计初衷是在于解决SelectionDAGISel在编译性能、可扩展性与复用性方面的痛点:SelectionDAGISel引入了专用中间表示并伴随较高的构建和合法化开销,而不同后端在DAG层的大量定制也导致维护成本上升\cite{Colombet2015GlobalISel}。

\par

除此之外,随着指令选择规则规模与复杂度增长,研究工作开始关注后端选择器相关的测试覆盖与缺陷发现,例如针对GlobalISel匹配表/选择路径的专门化测试与模糊测试。这些方向共同推动了指令选择从局部模式向可扩展、可验证、面向多目标优化的方向演进\cite{rong2024irfuzzer}。


2015年,苹果公司的Quentin Colombet提出:现有的指令选择框架SelectionDAGISel存在若干根本性局限,包括但不限于:编译速度缓慢、仅支持基本块级别的局部作用域、架构设计过于单一化等。多年来,开发者们投入了大量精力通过增加Target hooks和优化Pass来规避这些局限,但这些方案本身也带来了新的问题(如启发式算法精度不足、需提前预测指令选择器的执行行为等)与局限。他认为,当前已具备推出新一代指令选择框架GlobalISel的条件,该框架将从根源上解决上述问题,同时为提升代码生成质量创造新的可能\cite{Colombet2015GlobalISel}。次年,Quentin Colombet等人分享了GlobalISel在设计与实现层面取得的阶段性成果,同时明确后续仍需重点研发的技术方向\cite{Colombet2016GlobalISelDesign}。

\par

2017年3月,苹果公司的Justin Bogner指出其团队在指令选择测试中应用模糊测试与输入生成技术的实验过程及核心成果。深入探讨了在寻找高价值测试输入过程中的技术权衡,以及验证生成代码有效性的核心方案\cite{Bogner2017ISelFuzzing}。

\par

2019年,Daniel Sanders指出目前GlobalISel的开发重心主要集中在对各类目标架构的基础支持上,优化相关的工作投入相对有限\cite{Sanders2019GlobalISelCombiner}。近期,研发方向已转向优化能力的提升,目标是使其优化水平达到能够全面替代SelectionDAGISel的程度。并详细讲解Combiner的整体设计架构、支撑其运行的核心模块、它与 GlobalISel 其他组件的协同运作方式,以及该组件的测试和调试方法。

\par

2021年,Huang Zhufeng等人提出基于代价模型的指令合并优化以及全局寄存器组选择优化以及指令局部性优化等\cite{zhufeng2021optimization},有效解决了传统指令选择的全局优化能力缺失、编译开销大等问题,在申威平台实现了编译速度与代码质量的平衡。

\par

2022年,Kai Nacke等人分享了其为PowerPC目标架构实现GlobalISel框架的初步实践经验\cite{Nacke2022PowerPCGlobalISel},详细阐述了在适配过程中面临的架构特性适配挑战、与传统SelectionDAGISel后端的功能衔接方案,以及达成的阶段性成果(如核心指令集的无回退匹配、基础基准测试的编译通过率提升等),为其他架构的 GlobalISel 适配工作提供可复用的实践参考。

\par

2024年,有关GlobalISel的研究骤然增长,其中Pierre Houtryve提出为GlobalISel组合器基础设施新增输入/输出MIR模式支持\cite{Houtryve2024MIRPattern},该功能包含类PatFrag系统与类型推断机制,使开发者能够直接在TableGen中编写大量组合器规则;Tobias Stadler提出不再生成IR,而是直接输出GMIR,从而跳过代码生成流水线的首个转换阶段\cite{Stadler2024DirectGMIR}。在作者的应用场景中,该方案使编译速度提升了约 20\%;Jiahan Xie提出面向可扩展向量的突破性GlobalISel的实现方案\cite{Xie2024ScalableVectorGlobalISel},该方案以RISC-V向量扩展为目标场景。演讲深入探讨了支持可扩展向量算术逻辑单元及加载/存储指令过程中面临的核心挑战与创新解决方案;Madhur Amilkanthwar提出针对GlobalISel对部分指令和模式的支持不完善,导致其在该平台上需回退至传统的SelectionDAGISel的问题\cite{Amilkanthwar2024AArch64GlobalISel}。做出的核心贡献如下:通过在GlobalISel各编译阶段引入补丁,消除了其回退现象;同时对GlobalISel生成的代码进行了优化,显著缩小了其与SelectionDAGISel在AArch64高级SIMD平台上的性能差距。这些进展标志着GlobalISel框架的优化迈出了重要一步,使其向成为默认指令选择器的目标更近了一步。

\par

2025年,Nvidia的Neil Hickey提出:GlobalISel在所有后端的推广应用受到其指令选择场景覆盖不完全的限制,这导致其在部分场景下仍需回退至SelectionDAGISel\cite{Hickey2025GlobalISelCI}。为解决并监控这些局限性,Neil等人开发了一套专用的持续集成系统。该系统每日会自动构建最新版本的LLVM,通过编译一系列广泛的基准测试程序并记录回退事件,来为LLVM社区定位问题来源。


%*********************************************************************
% 1.3 论文研究内容和创新点
%*********************************************************************
\section{论文研究内容和创新点}


\subsection{研究内容}
结合前文分析可以看出,传统的SelectionDAGISel在实践中暴露出多方面的局限性。一方面,其实现依托于庞大的代码体系,且代码量会随功能的持续扩展不断膨胀,大幅提升了框架的开发与调试成本;另一方面,受限于SelectionDAG/SDNode的数据结构设计,指令选择过程需要频繁的进行内存分配与释放,这不仅压缩了编译优化策略的表达空间,也造成了编译时间的额外开销,影响整体编译效率。

\par

作为LLVM主推的SelectionDAGISel替代方案,GlobalISel凭借其模块化架构设计与良好的可扩展性,已被AArch64、AMDGPU、RISC-V以及X86等主流后端采用。然而在实际应用场景中,其生成的代码质量仍与成熟的SelectionDAGISel存在差距,尤其是在目标相关指令匹配与指令组合方面,GlobalISel仍有较大的优化空间。

\par

除此之外,DSP工具链现有CI测试流程存在着一些问题,主要体现在测试集来源单一和测试产物可读性与分析能力不足这两个问题。为解决上述问题,本文聚焦以下三方面研究:

\begin{enumerate}
	\item
	针对DSP硬件指令集以及寄存器布局等硬件特性,完成全局指令选择框架的定制化实现:通过扩展TableGen中的指令描述模板,补充DSP架构相关的指令语义与操作码映射关系,并结合DSP的硬件特点,设计了相应的寄存器组划分规则以及类型合法化逻辑,解决了通用GlobalISel在嵌入式DSP平台上适配性不足的问题。在此基础上,构建了适用于DSP后端的指令选择基础框架,替代原有的SelectionDAGISel,为后续指令选择优化与功能扩展提供了更具可维护性的实现基础。
	
	\item
	针对GlobalISel通用语义转换引入的指令冗余、硬件特性适配不充分等问题:分阶段设计并实现优化策略。在合法化前阶段,聚焦通用语义层面的冗余消除,减少通用转换带来的指令开销;在合法化后阶段,面向DSP指令集特性开展架构专属优化,提升生成代码的执行效率与存储密度。
	
	\item 
	针对现有测试体系存在的问题,设计并实现了一个与GitLab CI深度集成的性能测试平台:该平台在代码合并后自动触发测试任务,通过模拟器与芯片端执行测试,并将结果数据上传至Perf仓库。后端定时拉取数据并更新数据库,前端以可视化图表展示性能趋势。该方案支持多版本性能对比,并能精确追踪每次提交引入的性能变化,实现优化效果的持续量化评估。
	
\end{enumerate}


\subsection{创新点凝练}
针对传统指令选择方案在DSP架构下生成代码性能不足及测试体系不完善等核心问题,本文通过系统性的研究与工程实现,形成了兼具创新性与实用性的技术方案,主要创新点凝练如下:

\begin{enumerate}
	\item
	完成面向自研DSP架构的GlobalISel全流程定制化实现:针对传统SelectionDAGISel跨基本块优化弱、扩展性差、适配成本高的痛点,结合DSP寄存器架构、指令编码及专用运算单元特性,通过扩展TableGen模板、定制寄存器分配与类型合法化逻辑,实现GlobalISel从GMIR到机器码的端到端落地。该实现摒弃SelectionDAG专用中间表示的多层转换,以函数粒度保留全局优化视野,解决通用框架在DSP平台的适配难题,构建低耦合、易维护的编译器后端架构,降低后续优化与架构升级成本。
	
	\item
	设计分阶段指令选择优化体系:基于GlobalISel编译流程特性,提出合法化前通用优化+合法化后目标优化的分层优化策略。合法化前通过内存操作内联、冗余扩展/截断消除等策略,消除通用语义转换带来的指令冗余;合法化后针对DSP指令集特性,实现立即数装载精准适配、常量乘法强度削减等架构专属优化,充分挖掘硬件专用运算单元潜力,在保证编译效率的同时,显著提升生成代码的执行性能与存储密度。
	
	\item 
	构建全流程自动化测试评估平台:针对现有测试体系覆盖不全、性能分析能力薄弱的问题,设计并实现了支持随机测试生成与多维度性能评估的一体化平台。通过定制化扩展YARPGen工具,引入DSP专属内建函数与调试机制,生成覆盖复杂场景的随机测试用例;同时集成自动化性能采集、跨版本对比与可视化分析功能,支持代码体积、执行周期、指令打包效率等指标的量化评估,实现优化效果的持续追踪与缺陷快速定位,为编译器迭代提供系统化支撑。
	
\end{enumerate}


%*********************************************************************
% 1.4 论文组织结构
%*********************************************************************
\section{论文组织结构}
本文一共分为六个章节,围绕全局指令选择的实现与优化方法以及性能评估平台的设计与实现展开研究,论文整体结构安排如下。

\par

第一章为绪论部分,介绍了论文的研究背景与意义,重点分析了DSP架构与编译器技术的发展现状,梳理了指令选择技术,尤其是全局指令选择相关研究的国内外研究现状与存在的问题,在此基础上明确了本文的研究内容、技术路线和主要创新点。

\par

第二章为全局指令选择框架的实现部分。本章先对传统指令选择方案及其局限性进行了分析,并给出全局指令选择的基本概念和设计思路,之后分节详细阐述了面向DSP的全局指令选择的GMIR的生成、指令合法化、寄存器组选择以及机器指令选择这四个核心阶段的设计与实现。

\par

第三章为全局指令选择的优化策略的实现部分。本章先结合DSP架构特点从理论角度分析了全局指令选择优化的必要性和可行性,之后针对内存操作指令、乘法指令等典型应用场景,提出并实现了多种优化策略,提升了生成代码的执行效率和质量。

\par

第四章为测试评估平台的设计与实现部分。本章根据现有DSP编译器测试流程中存在的问题,确定了测试体系优化的需求与技术路径,之后从系统架构层面给出了两个子系统的整体设计方案,并在此基础上实现。

\par

第五章为实验和展示部分。实验部分从编译时间、代码体积和执行周期等多个维度来分别对全局指令选择实现的正确性及优化的有效性进行了验证,展示部分则通过运行截图来直观地展示测试评估平台的实现情况。

\par

第六章为总结与展望部分。本章对全文的研究工作进行了回顾,总结主要成果和技术创新,并对存在的不足与未来改进方向进行讨论。